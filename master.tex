%%%%%%%%%%%%%%%%%%%%%%%%%%%%%%%%%%%%%%%%%%%%%%%%%%%%%%%%
% Author: Vignesh Iyer                                 %
% MS CSE ASU                                           %
%%%%%%%%%%%%%%%%%%%%%%%%%%%%%%%%%%%%%%%%%%%%%%%%%%%%%%%%
% Modified by Ashkan Arabi

\documentclass{resume} % Use the custom resume.cls style
\usepackage{hyperref}
\RequirePackage{fontawesome}
%\hypersetup{
%	colorlinks=true,
%	linkcolor=blue,
%	urlcolor=cyan
%}

\begin{document}

\introduction[
    fullname=Ashkan Arabi,
    phone=(915) 888 - 9801,
    email=aarabimian@miners.utep.edu,
    linkedin=linkedin.com/in/ashkan-arabi,
    % github=ashkan.zone,
    github=github.com/AshkanArabim,
]

% \summary{Senior mechanical engineering student with internship experience in medical device manufacturing and product development. Project experience includes applications of software and hardware. Seeking full-time position May 2020 in medical device manufacturing and  pharmaceutical production. }

\education{
\educationItem[
university=University of Texas at El Paso{,} El Paso{,} TX,
graduation=May 2026,
grade=3.91,
majorgrade=4.00,
program=Bachelor of Science in Computer Science{,} Mathematics Minor,
coursework=Data Structures \& Algorithms{,} Discrete Math{,} Matrix Algebra{,} Computer Architecture{,} Statistics,
honors=Dean's List since Fall '22 {,} Houston Endowment Scholarship Recipient,
% Undergraduate Fellows{,} 
% 
]
}

\skills{
\skillItem[
category=Programming / Markup Languages,
skills=Python{,} HTML/CSS{,} C++{,} JavaScript{,} Java{,} C{,} LaTeX{,} Bash
]
\\
\skillItem[
category=Frameworks \& Software,
% for data science / ML:
skills=Linux{,} NumPy{,} Pandas{,} TensorFlow{,} git{,} React.JS{,} Conda{,} Docker{,} Microsoft Office,
% Other stuff:
% skills=Adobe Photoshop{,} Adobe Illustrator{,},
% mah:
% Django {,} Microsoft Office {,} SSH{,} Singularity{,} Webpack {,} Jupyter, Node.js
% DON'T INCLUDE:
% Node.JS, Webpack
]
\\
\skillItem[
    category=Certifications,
    skills=Coursera Deep Learning Specialization {-} October 2023
]
% \\
% \skillItem[
%     category=Foreign Languages,
%     skills=Persian (Native){,} Spanish (Fluent){,} French (Beginner)
% ]
}

%%%%%%%%%%%
% Accomplished [X] as measured by [Y], by doing [Z]
%%%%%%%%%%%

\begin{workSection}{Experience}
    \experienceItem[
        company=UTEP,
        location=El Paso{,} TX,
        position=Volunteer Researcher,
        duration=August 2023 {-} Present
    ]
    \begin{itemize}
        \vspace{-0.5em}
        \itemsep -6pt {}
        \item Collaborated to develop dynamic software for training users' weaknesses in spotting Phishing emails.
        \item Trained deep transformer model using \textbf{TensorFlow and Keras} to classify email cues, such as a sense or urgency.
        \item Preprocessed and visualized more than 5 datasets using \textbf{Pandas and NLTK} to use as training material for model.
    \end{itemize}
    \experienceItem[
        company=AI4ALL,
        location=Remote,
        position=Student Coordinator,
        duration=August 2023 {-} Present
    ]
    \begin{itemize}
        \vspace{-0.5em}
        \itemsep -6pt {}
        \item Mentored 16 students across 4 project groups in AI4ALL's Apply AI program.
        \item Helped students understand AI/ML concepts such as loss, backpropagation, CNNs, RNNs, transfer learning, etc.
    \end{itemize}
    % \experienceItem[
    %     company=UTEP,
    %     location=El Paso{,} TX,
    %     position=Tech Support Staff,
    %     duration=January 2023 {-} Present
    % ]
    % \begin{itemize}
    %     \vspace{-0.5em}
    %     \itemsep -6pt {}
    %     \item Provided top-tier hardware \& software tech support for 10+ students, staff, and faculty daily.
    %     \item Maintained and updated hardware \& software of 500+ computers through regular checkups.
    % \end{itemize}
    \experienceItem[
    company=Temple University,
    location=Philadelphia{,} PA,
    position=Undergraduate Researcher,
    duration=June {-} July 2023
    ]
    \begin{itemize}
        \vspace{-0.5em}
        \itemsep -6pt {}
        \item Conducted \href{https://drive.google.com/file/d/1HG4S5iSNb0nX2yN3LfSsRilemFDamnaN/view?usp=sharing}{\textbf{original research} \faExternalLink} in team of 2 about using Wi-Fi CSI for hand gesture recognition on smartphones.
        \item Developed CNN architecture to classify 5 gestures from 4 people in 6 scenarios using \textbf{Keras}. %(TensorFlow, Python, Bash)
        \item Obtained >90\% classification accuracy by using techniques such as LR Scheduling. %(TensorFlow, Keras)
        \item Used \textbf{Bash, Android Debug Bridge, and NumPy} for data extraction and preprocessing.
              % \item 
    \end{itemize}

\end{workSection}

\begin{workSection}{Projects}
    % \customItem[
    %     title=\href{https://github.com/AshkanArabim/email-classifier}{Adaptive Phishing Email Training System \faExternalLink},
    %     duration=August 2023 - Present,
    %     keyHighlight=Research project to develop dynamic software to train users' weaknesses in spotting phishing emails.
    % ]
    % \begin{itemize}
    %     \vspace{-0.5em}
    %     \itemsep -6pt {}
    %     \item Achieved 96\% phishing email detection accuracy to classify unlabeled emails for user training. (Keras, Python)
    % \end{itemize}

    % \customItem[
    %     title=Music Genre Classification,
    %     duration=Spring 2023,
    %     keyHighlight=Developed convolutional model to classify music genres based on their mel spectrogram.
    % ]
    % \begin{itemize}
    %     \vspace{-0.5em}
    %     \itemsep -6pt {}
    %     \item Led team of 3 to meet project deadlines by dividing tasks based on skill level (Project Management)
    %     \item Designed, tested, and implemented CNN model for classification (TensorFlow, Numpy, Conda)
    % \end{itemize}


    % \end{workSection}
    % \begin{workSection}{Personal Projects}


    % in case you change your mind: ----------------
    % \experienceItem[
    % company=Arizona State University,
    % location=Tempe{,} AZ,
    % position=Tutor (10 hours/week),
    % duration=Aug 2018 – May 2019
    % ]
    % \begin{itemize}
    %     \vspace{-0.5em}
    %     \itemsep -6pt {}
    %     \item Tutored 10-15 undergraduate engineering students per week in MATLAB programming and math coursework
    % \end{itemize}
    % ---------------------------------------------
    \customItem[
        title=\href{https://github.com/AshkanArabim/advent-of-code-2022}{Advent of Code 2022 - Annual Programming Challenge \faExternalLink},
        duration=Summer 2023,
        % keyHighlight=
    ]
    \begin{itemize}
        \vspace{-0.5em}
        \itemsep -6pt {}
        \item Coded \textbf{C++} solutions to 12 of 25 challenge questions using backtracking, and graph traversal algorithms.
        \item Used classes, queues, vectors, and streams to efficiently calculate results based on given inputs.
    \end{itemize}
    \customItem[
        title=\href{https://github.com/AshkanArabim/cybertweet-topics/tree/main}{Tweet topic detection \faExternalLink},
        duration=Summer 2023,
        % keyHighlight=
    ]
    \begin{itemize}
        \vspace{-0.5em}
        \itemsep -6pt {}
        \item Tuned \textbf{Keras} model to detect one of 8 cyber-security topics mentioned in Tweets.
        \item Utilized Twitter pre-trained word embeddings and Bidirectional GRU to achieve 93\% accuracy.
    \end{itemize}
    % \customItem[
    %     title=Tweet Sentiment Classification,
    %     duration=Summer 2023,
    %     keyHighlight=Built almost perfect model to classify positive and negative Tweets
    % ]
    % \begin{itemize}
    %     \vspace{-0.5em}
    %     \itemsep -6pt {}
    %     \item Achieved 99\% accuracy using standard NLP preprocessing and along with Embedding and LSTM layers. (Keras)
    % \end{itemize}
    \customItem[
        title=\href{https://ashkan.zone/}{Personal Portfolio \faExternalLink},
        duration=Spring 2023,
        % keyHighlight=Built complete portfolio website with React.JS and vanilla CSS. See https://ashkan.zone.
    ]
    \begin{itemize}
        \vspace{-0.5em}
        \itemsep -6pt {}
        % \item Designed and implemented simple responsive and interactive UI using only vanilla CSS. (HTML, CSS)
        \item Built portfolio website with \textbf{React.JS} components and custom vanilla \textbf{CSS}.
              % \item Used React.JS components to create single-page application, maximizing responsiveness. (React.JS, JavaScript)
    \end{itemize}
    % \customItem[
    %     title=Chess Knight Path Finder,
    %     duration=Spring 2023,
    %     keyHighlight=Used DFS graph traversal to find the most optimal path for chess knight.
    % ]
    % \begin{itemize}
    %     \vspace{-0.5em}
    %     \itemsep -6pt {}
    %     \item Given a starting and ending point, algorithm finds least number of steps to reach destination (JavaScript)
    % \end{itemize}
    % \customItem[
    %     title=NPM dom package...
    % ]
    % \customItem[
    %     title=Weather Web App,
    %     duration=Spring 2023,
    %     keyHighlight=Used variety of web APIs to develop weather forecast web app with dynamic UI elements
    % ]
    % \begin{itemize}
    %     \vspace{-0.5em}
    %     \itemsep -6pt {}
    %     \item Used asynchronous programming to fetch and render data from OpenWeatherMap.org (JavaScript)
    %     \item Bundled site assets using Webpack to minimize deployment size (Webpack)
    % \end{itemize}
    \customItem[
        title=\href{https://github.com/AshkanArabim/todolist}{To Do List Web App \faExternalLink},
        duration=Fall 2022,
        % keyHighlight=
    ]
    \begin{itemize}
        \vspace{-0.5em}
        \itemsep -6pt {}
        \item Developed interactive to-do list web app with local save function through vanilla \textbf{HTML{,} CSS \& JS}.
              % \item Developed all internal logic from scratch following OOP principles % (JavaScript)
        \item Used Chrome's local storage API for saving user data. % (JavaScript)
        \item Followed typical \textbf{git/GitHub} version control workflow during implementation. % (git, GitHub)
    \end{itemize}
    %     \customItem[
    %     title=\href{https://github.com/AshkanArabim/tic-tac-toe}{Tic Tac Toe Web App \faExternalLink},
    %     duration=Fall 2022,
    %     keyHighlight=Minimal Tic-Tac-Toe game with human-human{,} human-bot{,} and bot-bot game-modes.
    %     ]
    %     \begin{itemize}
    %         \vspace{-0.5em}
    %         \itemsep -6pt {}
    %         % \item Implemented smooth animations using CSS keyframes to improve site nativation. (CSS)
    %         \item Utilized CSS glow{,} shading{,} and animations for a modern and sleek look. (HTML, CSS)
    %         \item Followed OOP principles to track game progress. (JavaScript)
    %         \item Used typical git/GitHub workflow during implementation (git, GitHub)
    %     \end{itemize}

    \customItem[
        title=\href{https://github.com/AshkanArabim/blabber}{Blabber {-} a CLI Twitter replica \faExternalLink},
        duration=Fall 2022,
        % keyHighlight=Made CLI app that allows users to create an account{,} post{,} follow others{,} see a timeline{,} and delete their account
    ]
    \begin{itemize}
        \vspace{-0.5em}
        \itemsep -6pt {}
        \item Made \textbf{Java} CLI app for users to create an account{,} post{,} follow others{,} see a timeline{,} and delete their account.
        \item Used scanners and writers to save{,} update{,} and delete user information and posts. % (Java)
    \end{itemize}

    % \customItem[
    %     title=Contribution to Monkeytype,
    %     duration=Spring 2021,
    %     keyHighlight=Added 3 levels of Persian tests to open source typing test website
    % ]
    % \begin{itemize}
    %     \vspace{-0.5em}
    %     \itemsep -6pt {}
    %     \item Developed script to extract and check words from large text bodies using Vajehyab.com dictionary API (Python)
    %     \item Uploaded more than 21,000 most used Persian words to site database (git, GitHub)
    % \end{itemize}
\end{workSection}

\begin{workSection}{Leadership}
    % \customItem[
    %     title=CS Student Council,
    %     %         keyHighlight=One of the founders{,} and current member of the Council{,} helping students voice their concerns to the administration.
    %     keyHighlight=Empowered the CS students to voice their suggestions by providing feedback channels and hosting town halls.,
    %     duration=Fall 2023 - Present
    % ]
    % \begin{itemize}
    %     \vspace{-0.5em}
    %     \itemsep -6pt {}
    %     % \item Drafted ...
    % \end{itemize}

    \customItem[
        title=Treasurer - Association for Computing Machinery at UTEP,
        %        keyHighlight=Multiple roles{,} including publicity officer \& treasurer,
        % keyHighlight=Promoted side-projects and research by hosting informational and technical workshops to 40+ students each semester.,
        duration=Fall 2022 - Present
    ]
    \begin{itemize}
        \vspace{-0.5em}
        \itemsep -6pt {}
        \item Promoted side-projects and research by hosting informational and technical workshops to 40+ students each semester.
              % \item Recruited 50+ members through public presentations. # bad
              % \item Engaged with the student body to recruit more than 50 students.
              % \item Organized and planned of more than 6 workshops sponsored by companies such as Google.
              % \item Presented multiple interactive workshops attended by org members on topics such as Node.JS.
              %        \item Promoted research and open-source contributions by hosting info sessions and technical workshops to 40+ students each semester.
    \end{itemize}
\end{workSection}

% \begin{workSection}{Publications}
% cite ACM Mobihoc paper
% \end{workSection}

\end{document}