%%%%%%%%%%%%%%%%%%%%%%%%%%%%%%%%%%%%%%%%%%%%%%%%%%%%%%%%
% Author: Vignesh Iyer                                 %
% MS CSE ASU                                           %
%%%%%%%%%%%%%%%%%%%%%%%%%%%%%%%%%%%%%%%%%%%%%%%%%%%%%%%%
% Modified by Ashkan Arabi

\documentclass{resume} % Use the custom resume.cls style
\usepackage{hyperref}
\RequirePackage{fontawesome}
%\hypersetup{
%	colorlinks=true,
%	linkcolor=blue,
%	urlcolor=cyan
%}

% disable hyphening
\hyphenpenalty=10000
\exhyphenpenalty=10000

\begin{document}

\introduction[
    fullname=Ashkan Arabi,
    phone=(915) 888 - 9801,
    email=aarabimian@miners.utep.edu,
    linkedin=/in/ashkan-arabi,
    % github=ashkan.zone,
    github=github.com/AshkanArabim,
    % github=\faGithub,
]

% \summary{Senior mechanical engineering student with internship experience in medical device manufacturing and product development. Project experience includes applications of software and hardware. Seeking full-time position May 2020 in medical device manufacturing and  pharmaceutical production. }

\begin{workSection}{Education}
	% \textbf{Master of Science, Artificial Intelligence} \hfill {December 2026} \\
	\textbf{Bachelor of Science, Computer Science} \hfill {December 2025} \\
	The University of Texas at El Paso \hfill GPA: 3.93
\end{workSection}

%%%%%%%%%%%
% xx Accomplished [X] as measured by [Y], by doing [Z] --> USELESS
% Accomplished [x] (metric) by doing [y]
%%%%%%%%%%%

\begin{workSection}{Experience}
	
	% \experienceItem[
	% company=Primer.com,
	% location=San Francisco{,} CA,
	% position=Incoming Software Engineering Intern,
	% duration= December 2025 - May 2026,
	% ]
	
	\experienceItem[
	company=Capital One,
	location=McLean{,} VA,
	position=Software Engineering Intern,
	duration= June 2025 - August 2025,
	]
	\begin{itemize}
		\vspace{-0.5em}
		\itemsep -6pt {}
		% note: plan b stuff:
		% \item Created data pipeline for fraud detection during payment processing using {Java}, {Spring Boot}, \& {Hibernate}.
		% \item Created fraud detection data pipeline using {Java}, {Spring Boot}, \& {Hibernate}, contributing to projected {23\% reduction in fraud}.
		% note: project 1:
		% note: the internal tool was CODI
		% \item Built 
		% % root-cause-analysis dashboard 
		% alerting pipeline 
		% to detect card service failures \& notify SREs, reducing incident-response time by 
		% % {61\%}. % note: pulled from the presentation
		% {93\%}. % note: reduction from 4 hours to 15 mins
		% \item Reduced incident-response time by {93\%} by building alerting pipeline for card service failure detection.
		\item Built \& deployed anomaly-detection pipeline for Capital One's card decisioning system, notifying SREs of malfunctioning ML models and reducing incident-response time \textbf{from 4+ hours to <15 mins}.
		\item Wrote highly optimized \textbf{Databricks SQL} \& evaluated warehouse sizes to minimize pipeline cost \& latency.
		% \item Partnered with Observe Inc. to provide product feedback on dashboarding and alerting features, influencing vendor roadmap.

		% \item Leveraged internal tools to monitor {Snowflake} data and spot performance anomalies.
		% note: project 2: meh. does anyone really care about a New Relic to Observe migration?
		% note: may want to add the fact that I used Gemini here
		% \item Migrated monitoring dashboards from {New Relic} and {Splunk} to {Observe}, reducing observability fragmentation.
		% \item Centralized observability by writing Python + Gemini script to move dashboards from New Relic to Observe. % note: plan b
		% note: data src for observe vs new relic: https://techcrunch.com/2023/10/05/observability-platform-observe-raises-50m-in-debt-launches-gen-ai-features/
		% \item Slashed team observability cost by >70\%, using Python to move New Relic dashboards to Observe. % note: python is plan b
		
		% note: project 3:
		% \item Streamlined developer experience {Java} library tracking data lineage across card authorization system.
		% note: technically, consumer lineage is more broad. "consumer" just means other teams can use the plugin.
		% note: you could also add Mockito here.
		% note: other tech used: 
		% \item Extended {Java} data lineage plugin to support consumer spending data, improving audit-readiness.
		% \item Extended {Java} data lineage plugin to support consumer spending data though test-driven development with {JUnit}.
		% \item Extended \textbf{Java}-based lineage plugin to log custom data transformations using a YAML specification.
		\item Extended \textbf{Java}-based plugin to log custom \textbf{Apache Spark} data lineage events using a YAML specification. % note: mild plan b
		\item Applied \textbf{test-driven development} using \textbf{JUnit} and \textbf{Mockito} to ensure robust, maintainable code.

		% note: fuck it. just doing plan B is easier.
		% \item Created plugin to log card data transformations using {Java} and {JUnit} test-driven development, improving audit-readiness.
		% \item Created Java logging plugin to add traceability to card data transformations. % note: too vague. too generic.

		% note: databricks sql optimization to resume. from 45 mins to 7 mins
		% note: ^^ but snowflake if you need plan b. the original snowflake was fine. only databricks needed tweaking.
	\end{itemize}
	
	% \experienceItem[
	% company=UTEP,
	% location=El Paso{,} TX,
	% position=Undergraduate Research Assistant,
	% duration= January 2025 - May 2025,
	% ]
	% \begin{itemize}
	% 	\vspace{-0.5em}
	% 	\itemsep -6pt {}
	% 	% 8+ might be considered "plan B"
	% 	% \item Provisioned {Docker Swarm} system to stream hardware usage information from {8+} compute nodes into one dashboard.
	% 	\item Provisioned {Docker Swarm} 
	% 	distributed 
	% 	system to monitor hardware information of 8-node GPU cluster.
	% 	% to stream hardware information 
	% 	% over the network
	% 	% from {8+} compute nodes into one dashboard.
	% 	\item Created and tested {Docker} containers for Nvidia GPU cluster to facilitate reproducible HPC research.
	% \end{itemize}
	
	% \experienceItem[
	% company=Texas Instruments,
	% location=Dallas{,} TX,
	% position=Information Technology Intern,
	% duration= May 2024 - August 2024,
	% ]
	% \begin{itemize}
	% 	\vspace{-0.5em}
	% 	\itemsep -6pt {}
	% 	% \item Maintained automation software for chip fabrication operations.
	% 	% \item Developed scripts for carrier robots in AutoShell (proprietary scripting language) to use in chip fabrication.
	% 	% \item Set up infrastructure for manufacturing automation by configuring and deploying {Docker} containers.
	% 	% note: 50% is plan b
	% 	% \item Set up {robotic automation} infrastructure by deploying {Docker} containers on {Linux} servers, reducing costs by up to {50\%}. 
	% 	\item Reduced production line costs by \textbf{$\sim$34\%} by deploying \textbf{Docker} infrastructure for wafer transporter robots.
	% 	% \item Set up infrastructure for {fab transporter robots} by configuring {Docker} containers on {Linux} servers.
	% 	% \item Ensured system reliability through unit and integration testing with Insomnia.
	% 	% \item Ensured system reliability through unit and integration testing with Insomnia before final deployment.
	% 	% \item Coordinated with {10+} sysadmins \& stakeholders to configure Oracle {SQL} databases and internal {APIs}.
	% 	\item Met with {10+} stakeholders to configure \textbf{Oracle SQL} databases, {REST APIs}, and documentation.
	% 	% \item Created Oracle SQL tables and views feeding 
	% \end{itemize}
	
	\experienceItem[
	company=GNOME Foundation,
	location=Remote,
	position=Open-Source Contributor,
	duration=December 2023 {-} June 2024
	]
	\begin{itemize}
		\vspace{-0.5em}
		\itemsep -6pt {}
		% note: commits and their status
		% - src: https://gitlab.gnome.org/GNOME/gnome-clocks/-/merge_requests/?sort=created_date&state=all&author_username=AshkanArabim&first_page_size=20
		%   - open: https://gitlab.gnome.org/GNOME/gnome-clocks/-/merge_requests/308
		%   - merged: https://gitlab.gnome.org/GNOME/gnome-clocks/-/merge_requests/309
		%   - open: https://gitlab.gnome.org/GNOME/gnome-clocks/-/merge_requests/314

		%		\item Improved the GNOME Clocks user experience by fixing various reported bugs.
		%	\item Contributed to the development of GNOME Clocks, an {app used by thousands of Linux users} to keep track of time.
		% \item (CHANGE THIS) learned Vala and GTK *on my own!!!!*. only using online docs and code examples
		% \item Contributed to GNOME Clocks; {used by thousands of Linux users} to track  time. % note: bloomberg said this is shit.
		\item Implemented timer editing and full-screen timers in GNOME Clocks, \textbf{used by 40+ million} to track time.
		\item Opened \textbf{3 PRs}, reviewed code with maintainers, and merged my \textbf{GTK + Vala} components.
		% features such as 
		% full-screen timers and timer editing using {Vala} bindings and {GTK} components \href{https://youtu.be/fDzYWusOLow}{({video demo })}. % (in progress)
		% full-screen timers and timer editing using {Vala} bindings and {GTK} components.
		% \item Solved timers not progressing during system suspend by revising timer logic. % (merged)
		% \item Learned completely new language and framework {in one week} before starting development.
		% \item Authored functionality for world clock to show country and state when two cities have the same name. % (open)
		%		\item Translated the GNOME Builder IDE to Persian. % <<- they seriously don't care. only use if you have space
		%		\item Expanded the accessibility of the GNOME Builder IDE by translating it to Persian.
		% \item Merged {2} pull requests that passed the CI/CD pipeline and were deployed to production. % 
		% \item Merged {2} pull requests that were deployed to production. % 
	\end{itemize}
	
	\experienceItem[
	company=UTEP Vision and Learning Lab,
	location=El Paso{,} TX,
	position=Undergraduate Research Assistant,
	duration=January 2024 - May 2024,
	]
	\begin{itemize}
		\vspace{-0.5em}
		\itemsep -6pt {}
		\item Contributed to creation of autistic vs neurotypical speech classifier by
		% using generative AI to 
		% synthesizing training data.
		creating synthetic training data.
%		\item Used {PyTorch} to reconstruct state-of-the-art networks proposed in papers.
		%  \item Developed an \href{https://github.com/AshkanArabim/accent-change-paper-implementation}{{accent-changer model} } able to convert foreign English accents to native using HuggingFace pretrained AI/ML models through their  API.
		% \item Developed \href{https://github.com/AshkanArabim/accent-change-paper-implementation}{accent-changer model } to convert foreign English accents to native using HuggingFace pretrained AI/ML models.
		% \item Built \href{https://github.com/AshkanArabim/accent-change-paper-implementation}{{accent-changer}} using {HuggingFace} pretrained models to change foreign English accents to native.
		\item Built tone-changer by fine-tuning \textbf{HuggingFace} models to change normal tone to autistic. % note: plan b
		% \item Implemented a \href{https://github.com/AshkanArabim/neural-style-transfer}{{neural style-transfer model} } in {PyTorch} to learn PyTorch \& explore usage for accent-changing.
		\item Implemented \href{https://github.com/AshkanArabim/neural-style-transfer}{{neural style-transfer model}} in \textbf{PyTorch} to explore usage for accent-changing.
%		\item Used Python's multiprocessing library to parallelize CPU-intensive experiments.
		% \item Reduced evaluation script runtime 
		% 	%{from 24+ hrs to 10 mins} 
		% 	{by 14400\%} 
		% 	by rewriting loops as higher-dimension tensor operations
		\item Increased evaluation script performance by
			%{from 24+ hrs to 10 mins} 
			\textbf{144x} 
			reimplementing KNN in PyTorch to utilize GPU.
	\end{itemize}
	
%	\experienceItem[
%	company=UTEP,
%	location=El Paso{,} TX,
%	position=Undergraduate TA for Data Structures \& Algorithms,
%	duration=January 2024 - present,
%	]
%	\begin{itemize}
%		\vspace{-0.5em}
%		\itemsep -6pt {}
%		\item Helped 30+ students understand challenging concepts such as B-Trees and Dijkstra's Algorithm.
%	\end{itemize}

%    \experienceItem[
	%        company=AI4ALL,
%        location=Remote,
%        position=Student Coordinator,
%        duration=August {-} December 2023
%    ]
%    \begin{itemize}
%        \vspace{-0.5em}
%        \itemsep -6pt {}
%        \item Mentored 16 students across 4 project groups in AI4ALL's Apply AI program.
%%        \item Helped students understand AI/ML concepts such as loss, backpropagation, CNNs, RNNs, reinforcement learning, etc.
%        \item Helped students understand concepts such as loss functions, optimization, neural networks, CNNs,
%%         RNNs, 
%        reinforcement learning, etc.
%    \end{itemize}
    % \experienceItem[
    %     company=UTEP,
    %     location=El Paso{,} TX,
    %     position=Tech Support Staff,
    %     duration=January 2023 {-} Present
    % ]
    % \begin{itemize}
    %     \vspace{-0.5em}
    %     \itemsep -6pt {}
    %     \item Provided top-tier hardware \& software tech support for 10+ students, staff, and faculty daily.
    %     \item Maintained and updated hardware \& software of 500+ computers through regular checkups.
%     \end{itemize}

	% \experienceItem[
	% company=UTEP,
	% location=El Paso{,} TX,
	% position=Volunteer Researcher,
	% duration=August 2023 {-} October 2023
	% ]
	% \begin{itemize}
	% 	\vspace{-0.5em}
	% 	\itemsep -6pt {}
	% 	\item Contributed to the development of software for training users' weaknesses in spotting Phishing emails.
	% 	% \item Classified emails across 7 phishing attributes 
	% 	% % (e.g. sender\_mismatch, suspicious\_subject) 
	% 	% with {78\%} accuracy by developing multi-output NLP classification model.
	% 	\item Classified emails across {7} phishing attributes with {78\%} accuracy by designing {TensorFlow} NLP machine learning model.
	% 	% \item Trained deep transformer model using {Keras} to classify email cues, such as a sense of urgency.
	% 	\item Preprocessed and visualized more than 5 datasets using {Python} and {NLTK} to use as training material for model.
	% \end{itemize}

    \experienceItem[
    company=Temple University,
    location=Philadelphia{,} PA,
    position=Research Intern,
    duration=June 2023 {-} July 2023
    ]
    \begin{itemize}
        \vspace{-0.5em}
        \itemsep -6pt {}
				\item Published
				\href{https://dl.acm.org/doi/10.1145/3565287.3617613}{{paper} } 
				in \textbf{ACM MobiHoc} about using Wi-Fi CSI for hand gesture recognition on smartphones.
        % \item Wrote
        % \href{https://dl.acm.org/doi/10.1145/3565287.3617613}{{first-author publication }}{,} 
        % {accepted into ACM MobiHoc '23}{,}
				% about using Wi-Fi 
				% % CSI 
				% for hand gesture recognition on phones.
        % \item Developed CNN architecture to classify 5 gestures from 4 people in 6 scenarios using {Keras}. %(TensorFlow, Python, Bash)
        \item Developed CNN model to classify 5 gestures 
				% from 4 people
				in 6 scenarios using \textbf{TensorFlow} with \textbf{>90\% accuracy}. %(TensorFlow, Python, Bash)
%        \item Obtained >90\% classification accuracy by using techniques such as LR Scheduling.
        % \item {1st place} for the best REU site final presentation. %(TensorFlow, Keras)
%        \item Used {Bash, Android Debug Bridge, and NumPy} for data extraction and preprocessing.
              % \item 
        % \item Selected to receive the Student Travel Grant Award of {\$1,200} out of hundreds of researchers to cover travel expenses.
        % \item Selected to receive travel grant of {\$1,200} out of hundreds of researchers to cover travel expenses.
				% note: src: email titled "RMBL REU Award Letter - Arabi". They gave the grant to almost everyone who had done an REU though.
    \end{itemize}

\end{workSection}

\begin{workSection}{Projects}

	\customItem[
		title=Docker Exploit Manager,
		technologies= | JavaScript{,} Svelte{,} Python{,} Docker{,} Neo4j{,} Trivy{,} Grype{,} Dockle{,} Metasploit,
		% duration=Solo
	]
	\begin{itemize}
		\vspace{-0.5em}
		\itemsep -6pt {}
		\item Created full-stack application to streamline Docker container exploit assessments for DEVCOM testers.
		\item Automated container security scans with \textbf{Trivy, Grype, Dockle} to enumerate CVE-referenced vulnerabilities.
		\item Developed interactive topology map using \textbf{Docker API}, \textbf{Nmap}, \& \textbf{Neo4j} for container discovery.
		% \item Implemented secure SSH tunnel connectivity with automated retries within 10 seconds.
		\item Orchestrated automated penetration tests using \textbf{Metasploit} to identify and validate exploitable flaws.
	\end{itemize}
	
	% \customItem[
	% title=Ashdocs (in progress),
	% technologies= | Go(Golang){,} Websockets,
	% % the repo is private rn: https://github.com/AshkanArabim/athlytix
	% % duration=%Solo | 
	% % August 2025 - present,
	% ]
	% \begin{itemize}
	% 	\vspace{-0.5em}
	% 	\itemsep -6pt {}
	% 	\item Created Google Docs clone in {Go}, able to handle multiple users' parallel real-time edits on documents.
	% \end{itemize}
	
	\customItem[
	title=\href{https://github.com/AshkanArabim/hackerhunt}{hackerhunt.tech \faExternalLink},
	technologies= | Python{,} Django{,} React{,} TypeScript{,} Nginx{,} Docker{,} AWS SES,
	duration=%Solo | 
	% January 2025 - May 2025,
	]
	\begin{itemize}
		\vspace{-0.5em}
		\itemsep -6pt {}
		\item Democratized collaboration by making a platform for students to find collaborators for technical projects. 
		% \item Used {Django + ChakraUI + Redux + React-Router + TypeScript} to create fast, state-of-the-art fullstack webapp.
		% \item Used {Django REST Framework} (Service-Oriented Architecture) \& {Django ORM} to create MVP webapp.
		% \item Incorporated email verification, password resetting, and transactional notifications through {AWS SES}.
		\item Minimized costs by self-hosting HTTPS page using \textbf{Nginx}, \textbf{Docker Compose}, and manual DNS setup.
		% \item token-based authentication????
		% \item Raised {\$1000 in pre-seed funding} for first round of market validation advertisements, helping to get {30+ beta testers}. % 30 users is plan b
		\item Raised \textbf{\$1000 in pre-seed funding} for market validation advertisements, helping to get \textbf{30+ beta testers}. % 30 users is plan b
		% \item Created Bash {CI/CD} script to immediately deploy new releases.
		% \item TODO: add hard numbers after making it public
	\end{itemize}
	
	% \customItem[
	% title=Athlytix,
	% % the repo is private rn: https://github.com/AshkanArabim/athlytix
	% duration=Team of 4 | April 2025 - May 2025,
	% ]
	% \begin{itemize}
	% 	\vspace{-0.5em}
	% 	\itemsep -6pt {}
	% 	\item Created platform to connect MMA fighters seeking improvement with gyms/orgs seeking talent, using {Supabase} and {React}.
	% 	% \item Enabled scalable extraction of statistics from training \& fighting videos by using {Deno serverless functions}.
	% 	\item Won the {Innovator Award} at STTE foundation's AI hackathon.
	% \end{itemize}

	% \customItem[
	% title=\href{https://github.com/chesterCaii/back-logz/}{Backlogz }, % HackWesTX
	% technologies= | React Native{,} Expo-Router{,} TypeScript
	% % duration=Team of 4 | April 2025,
	% ]
	% \begin{itemize}
	% 	\vspace{-0.5em}
	% 	\itemsep -6pt {}
	% 	% \item Boosted personal productivity by creating an app to generate 90-second podcasts based on users' backlog of interesting topics.
	% 	% \item Created Android \& iOS app to generate 90-second podcasts from users' backlog of interesting topics.
	% 	% \item Shipped app for {Android, iOS, and web} by using {React Native} and {Expo-Router} cross-platform frameworks.
	% 	% \item Won {3rd place} in SFHacks 2025's startup track.
	% 	\item Won 3rd in SFHacks 2025 by creating app to generate 90 sec podcasts from users' research backlog.
	% \end{itemize}
	
	% \customItem[
	% % source code in class files from spring 25
	% title=Video Stabilizer,
	% duration=Solo | April 2025,
	% ]
	% \begin{itemize}
	% 	\vspace{-0.5em}
	% 	\itemsep -6pt {}
	% 	\item Created a video stabilizer using {OpenCV}, ORB feature matching, and {RANSAC} to reduce camera shake in real-time footage.
	% \end{itemize}
	
	% \customItem[
	% title=Astrophotography Reconstruction,
	% duration=February 2025,
	% ]
	% \begin{itemize}
	% 	\vspace{-0.5em}
	% 	\itemsep -6pt {}
	% 	\item Reconstructed high-quality astrophotography from 1000 noisy frames using {lucky imaging}, edge-based sharpness ranking, and intensity correction in {NumPy/SciPy}.
	% \end{itemize}
	
	% \customItem[
	% title=Virtual Ad,
	% duration=February 2025,
	% ]
	% \begin{itemize}
	% 	\vspace{-0.5em}
	% 	\itemsep -6pt {}
	% 	\item Built a virtual ad placement system for sports broadcasting using {OpenCV}, {NumPy}, and {SKImage}.
	% \end{itemize}

	% \customItem[
	% title=PanoCam,
	% duration=February 2025,
	% ]
	% \begin{itemize}
	% 	\vspace{-0.5em}
	% 	\itemsep -6pt {}
	% 	\item Created an image stitching pipeline using {OpenCV and SKImage}, capable of combining many small images into one large image.
	% \end{itemize}

	%%%%%%%%%%
	% note: there was a bloomberg tech lab project that used Redis somewhere,
	% but I'm not including it because the rest was just React, Flask, etc.
	% really boring tech, and we didn't even finish the damn project.
	%%%%%%%%%%

	% \customItem[
	% title=PALADIN,
	% duration=February 2025 - April 2025,
	% ]
	% \begin{itemize}
	% 	\vspace{-0.5em}
	% 	\itemsep -6pt {}
	%   TODO: add the "problem solving for ai" project if you feel like it. I personally winged it so idk if it should be here.
	% \end{itemize}

	% \customItem[
	% title=\href{https://github.com/AshkanArabim/flirtify}{Flirtify },
	% duration=Solo | December 2024 - January 2025,
	% ]
	% \begin{itemize}
	% 	\vspace{-0.5em}
	% 	\itemsep -6pt {}
	% 	% \item Created an Android chat app that uses Google's {Gemini API} to add a flirting tone to all sent messages.
	% 	\item Created an Android chat app that uses {Google Cloud}'s Gemini API to add a flirting tone to all sent messages.
	% 	\item Used {Flutter (Dart)} to create the mobile UI and {Firebase} to enable instant communication.
	% \end{itemize}

	% title=\href{https://github.com/AshkanArabim/news-briefer}{News Bridge - BorderHack 2024 },
	% duration=Team of 4 | September 2024,
	% original title ^^
	\customItem[
		title=\href{https://github.com/AshkanArabim/newsbridge}{Newsbridge (news.ashkan.zone) \faExternalLink},
		technologies= | FastAPI{,} React{,} PostgreSQL{,} JavaScript{,} Docker,
		duration=%Team of 4 | 
		% September 2024 - December 2024,
	]
	\begin{itemize}
		\vspace{-0.5em}
		\itemsep -6pt {}
		% \item Increased access to diverse news by creating app for daily briefings from RSS feeds, regardless of source language.
		% \item Increased access to diverse news by creating an AI RSS news translator \& orator with {55 GitHub stars}.
		\item Increased access to diverse news by creating an AI RSS news translator \& orator with \textbf{55 GitHub stars}.
		% \item Implemented a microservice architecture using {Docker Compose}, for future load-balancing \& increased capacity.
		\item Ensured load-balancing \& scaling capacity by using a \textbf{Docker Compose microservice architecture}.
		% \item Enabled immediate playback using {Python async programming} to generate and play one sentence at a time.
%		\item Developed app that takes RSS news feeds (from any source in any language), translates \& summarizes top stories, and plays the summary back like a news briefing podcast.
		% \item Used {Flask} with {PostgreSQL} to fetch users' news sources and pass them to LLM and TTS models for summarization.
		% \item Used {FastAPI} with {PostgreSQL} to fetch users' news sources and pass them to {LLM and TTS models} for summarization.
		% \item Used {Flask} with {PostgreSQL} hosted on {Google Cloud} to store users' native language \& news sources.
		% \item Loaded users' customized UI using {Redux.JS} and {React.JS}.
		% \item Programmed interactive web UI using {Redux.JS} and {React.JS}.
		% (not including other parts of project cuz I didn't make them)
		% \item Developed and tested {REST API interfaces} for team members to use in the front-end.
	\end{itemize}
	
	% \customItem[
	% title=\href{https://devpost.com/software/vocowbulary-courses}{Vocabulary Courses }, % HackWesTX
	% duration=Team of 4 | September 2024,
	% ]
	% \begin{itemize}
	% 	\vspace{-0.5em}
	% 	\itemsep -6pt {}
	% 	\item Built full-stack web-app to help non-native speakers practice their pronunciation using spaced repetition. % (team of 4, 24 hours)
	% 	\item Conceived custom scheduling algorithm on {Node.JS} and {MongoDB} (noSQL) back-end to fetch the next practice word.
	% 	% \item Developed and tested {REST API interfaces} for team members to use in the front-end. % note: bloomberg said this sounds redundant
	% \end{itemize}
	
%	\customItem[
%			title=\href{https://github.com/AshkanArabim/forced-flirtation}{Forced Flirtation (in progress) - full-stack chat app with a twist },
%			duration=August - September 2024,
%	]
%	\begin{itemize}
%			\vspace{-0.5em}
%			\itemsep -6pt {}
%			\item Created a chat app that uses OpenAI's ChatGPT API to add a flirting tone to all messages users send on the platform.
%			\item Used a {Node.JS} backend, paired with {MongoDB} and {Redux} to enable instant communication.
%	\end{itemize}
	
	% \customItem[
	% 	title=\href{https://blog.ashkan.zone/}{Personal Blog },
	% 	duration=Solo | August 2024,
	% ]
	% \begin{itemize}
	% 	\vspace{-0.5em}
	% 	\itemsep -6pt {}
	% 	% \item Created a blog to share useful programming and life tips.
	% 	% \item Used {React.JS}, {TypeScript}, and {Tailwind CSS} to quickly bring my design to life, and {Gatsby} to add Markdown support.
	% 	% \item Used {React.JS}, {TypeScript}, and {TailwindCSS} to develop blog UI.
	% 	% \item Added Markdown support with {Gatsby}, utilizing {GraphQL} to fetch blog entries.
	% 	\item Created personal blog with {React.JS} \& {TypeScript}, and used {GraphQL} to fetch blog entries. % tailored for Meta, who doesn't care about abstract frameworks
	% \end{itemize}
	
	% \customItem[
	% 	title=\href{https://github.com/AshkanArabim/pwaang-extended}{PWAANG - board game written in Haskell },
	% 	duration=Solo | May 2024,
	% 	% keyHighlight=
	% ]
	% \begin{itemize}
	% 	\vspace{-0.5em}
	% 	\itemsep -6pt {}
	% 	%		\item Implemented a board game entirely in {Haskell}, using monads, pattern matching, and recursion.
	% 	% \item Implemented a board game in {Haskell} (functional programming language) using monads, pattern matching, and recursion.
	% 	\item Implemented a full-fledged board game in {Haskell}, pushing myself to learn functional programming.
	% \end{itemize}

	% \customItem[
	% 	title=Bloomberg Tech Lab on Campus,
	% 	duration=Team of 2 | April 2024,
	% ]
	% \begin{itemize}
	% 	\vspace{-0.5em}
	% 	\itemsep -6pt {}
	% 	\item {One of 40} students selected to collaborate with Bloomberg engineers 
	% 	% in a small group setting 
	% 	to build stock data processing application. % (official)
	% 	% \item Utilized {Python} to design and implement a robust message queue system using {RabbitMQ}, enhancing real-time data processing and communication between producer and consumer components. % (official)
	% 	\item Designed and implemented a queued messaging system using {RabbitMQ} for real-time data stream processing.
	% 	% \item Designed and implemented a message passing system using {RabbitMQ} for real-time data stream processing.
	% 	% \item Developed a deeper understanding of core Python/CS concepts (Classes, Inheritance, OOP), as well as financial domain knowledge (Tickers, Industry Sectors). % (official)
	% \end{itemize}
	
%	\customItem[
%		title=\href{https://github.com/AshkanArabim/bombshell}{Bombshell - Unix shell written in Python },
%		duration=April 2024,
%		% keyHighlight=
%	]
%	\begin{itemize}
%		\vspace{-0.5em}
%		\itemsep -6pt {}
%		\item Added features like pipes, redirection, and background tasks by using multi-threading and semaphores, ensuring concurrency.
%	\end{itemize}
	
	% \customItem[
	% 	title=\href{https://github.com/AshkanArabim/os-file-transfer}{TCP/IP file-transfer },
	% 	duration=Solo | March 2024,
	% 	% keyHighlight=
	% ]
	% \begin{itemize}
	% 	\vspace{-0.5em}
	% 	\itemsep -6pt {}
	% 	\item Built custom client-server system to stream files over basic {TCP sockets} from {Python}'s "socket" networking library.
	% 	% \item Enabled parallel transfers using forked child processes for each connection.
	% 	% \item Handled file metadata with {struct-based binary headers}, manually reassembled byte streams on server.
	% \end{itemize}
	
% 	\customItem[
% 		title=\href{https://github.com/AshkanArabim/OOP-project-1}{CLI Car Dealership },
% 		duration=Team of 3 | April 2024,
% 		% keyHighlight=
% 	]
% 	\begin{itemize}
% 		\vspace{-0.5em}
% 		\itemsep -6pt {}
% %		\item Wrote an Object-Oriented car dealership software in {Java}, using inheritance, polymorphism, and interfaces.
% 		\item Wrote an Object-Oriented car dealership software in {Java} following MVC architecture \& object oriented design patterns. % with 2300+ lines of code.
% 		\item Used {Git \& GitHub} features such as pull requests, merges and branches to work in team of 3.
% %		\item Used Git \& GitHub features such as pull requests, merges, branches, and rebasing for teamwork in group of 3.
% %		\item Drafted overall software design in UML use-case and class diagrams.
% %		\item Followed Agile methodology to design, plan, and develop three phases of feature additions.
% %		\item Included operations to buy{,} restock{,} or add cars{,} monitor the revenue of each vehicle type{,} add / remove users, etc.
% %		\item Included automatic
% 	\end{itemize}
	
%	\customItem[
%		title=\href{https://github.com/AshkanArabim/gtk-timer}{Linux Timer },
%		duration=January 2024,
%		% keyHighlight=
%	]
%	\begin{itemize}
%		\vspace{-0.5em}
%		\itemsep -6pt {}
%		\item Developed a {Linux} timer application using {GTK \& Vala}, using an event-driven architecture.
%		\item Implemented functionalities for starting, pausing, resetting, and editing timers, using different GTK4 widgets \& signals.
%	\end{itemize}
	
% 	\customItem[
% 		title=\href{https://github.com/AshkanArabim/pong-msp430}{Pong for MSP430 \faExternalLink},
% 		technologies=| C
% 		% duration=Solo | November 2023,
% 		% keyHighlight=
% 	]
% 	\begin{itemize}
% 		\vspace{-0.5em}
% 		\itemsep -6pt {}
% 		%		\item Implemented smooth graphics using interrupts to avoid graphical bottlenecks.
% 		% \item Designed and implemented a {C} Pong game for MSP430, with paddle movement, ball physics, and score tracking.
% 		\item Designed and implemented Pong for MSP430, with paddle movement, ball physics, and score tracking.
% %		\item Implemented interrupt-driven input handling for buttons, ensuring responsive gameplay with 30+ FPS.
% 		\item Achieved {30+ FPS} gameplay by using partial framebuffer updating instead of redrawing whole screen.
% 		\item Integrated buzzer audio feedback for game events such as ball-wall collisions and score updates.
% 	\end{itemize}
	
%   vv not including hacktoberfest 2023 because it was mostly wasted on translation projects...
	
%	\customItem[
%	title=Hacktoberfest 2023,
%	duration=October 2023,
%	% keyHighlight=
%	]
%	\begin{itemize}
%		\vspace{-0.5em}
%		\itemsep -6pt {}
%		\item Contributed to three open-source repositories through bug-fixes and translations.
%		\item Fixed subtitle cutoff bug in ASCII video player written in {C} by debugging the subtitle buffer. \href{https://github.com/aidancrowther/ASCIIPlay}{}
%	\end{itemize}

	% \customItem[
	% title=\href{https://github.com/AshkanArabim/advent-of-code-2022}{Advent of Code 2022 - Annual Programming Challenge },
	% duration=Solo | August 2023,
	% % keyHighlight=
	% ]
	% \begin{itemize}
	% 	\vspace{-0.5em}
	% 	\itemsep -6pt {}
	% 	% \item Coded {C++} solutions to 12 of 25 challenge questions using backtracking, graph traversal algorithms, and more.
	% 	\item Coded {C++} solutions to 12 of 25 data structures \& algorithms challenges.
	% 	\item Used classes, queues, vectors, and streams to efficiently calculate results based on given inputs.
	% \end{itemize}

    % \customItem[
    %     title=\href{https://github.com/AshkanArabim/email-classifier}{Adaptive Phishing Email Training System },
    %     duration=August 2023 - Present,
    %     keyHighlight=Research project to develop dynamic software to train users' weaknesses in spotting phishing emails.
    % ]
    % \begin{itemize}
    %     \vspace{-0.5em}
    %     \itemsep -6pt {}
    %     \item Achieved 96\% phishing email detection accuracy to classify unlabeled emails for user training. (Keras, Python)
    % \end{itemize}

    % \customItem[
    %     title=Music Genre Classification,
    %     duration=Spring 2023,
    %     keyHighlight=Developed convolutional model to classify music genres based on their mel spectrogram.
    % ]
    % \begin{itemize}
    %     \vspace{-0.5em}
    %     \itemsep -6pt {}
    %     \item Led team of 3 to meet project deadlines by dividing tasks based on skill level (Project Management)
    %     \item Designed, tested, and implemented CNN model for classification (TensorFlow, Numpy, Conda)
    % % in case you need more:
    %% original linkedin title: AI4ALL College Pathways Participant
    % %- Led team of 3 to meet project deadlines by dividing tasks based on skill level.
    %%- Developed convolutional model to classify music genres based on their Mel spectrogram.
    %% - Learned about the applications and fundamental technical concepts of AI including the types of machine learning techniques and neural networks. 
    %% - Investigated ethical implications related to data processing and AI implementation 
    %% - Collaborated with peers on a project to preditct sudents' academic performance in college based on their background.
    % \end{itemize}

    % sample experience item: ----------------
    % \experienceItem[
    % company=Arizona State University,
    % location=Tempe{,} AZ,
    % position=Tutor (10 hours/week),
    % duration=Aug 2018 – May 2019
    % ]
    % \begin{itemize}
    %     \vspace{-0.5em}
    %     \itemsep -6pt {}
    %     \item Tutored 10-15 undergraduate engineering students per week in MATLAB programming and math coursework
    % \end{itemize}
    % ---------------------------------------------
    
%    \customItem[
%        title=\href{https://github.com/AshkanArabim/cybertweet-topics/tree/main}{Tweet topic detection },
%        duration=August 2023,
%        % keyHighlight=
%    ]
%    \begin{itemize}
%        \vspace{-0.5em}
%        \itemsep -6pt {}
%        \item Tuned {Keras} model to detect one of 8 cyber-security topics mentioned in Tweets.
%        \item Utilized Twitter pre-trained word embeddings and Bidirectional GRU to achieve 93\% accuracy.
%    \end{itemize}
    
    % \customItem[
    %     title=Tweet Sentiment Classification,
    %     duration=Summer 2023,
    %     keyHighlight=Built almost perfect model to classify positive and negative Tweets
    % ]
    % \begin{itemize}
    %     \vspace{-0.5em}
    %     \itemsep -6pt {}
    %     \item Achieved 99\% accuracy using standard NLP preprocessing and along with Embedding and LSTM layers. (Keras)
    % \end{itemize}

	% \customItem[
	% 	title=Home Server,
	% 	duration=Solo | August 2023 - February 2024, % these dates aren't exact
	% 	% keyHighlight=
	% ]
	% \begin{itemize}
	% 	\vspace{-0.5em}
	% 	\itemsep -6pt {}
	% 	\item Built a personal {Ubuntu Server} to use for personal backups, self-hosting, file sharing, and GPU workloads.
	% 	\item Configured {12TB RAID5 array} with redundancy for reliable data backup and archiving. % with rsync?
	% 	% \item Created remote-accessible ML environment using an RTX 3060 GPU and 64GB RAM to use for research and {self-hosted LLMs}.
	% \end{itemize}
    
%    \customItem[
%        title=\href{https://ashkan.zone/}{Personal Portfolio },
%        duration=Spring 2023,
%        % keyHighlight=Built complete portfolio website with React.JS and vanilla CSS. See https://ashkan.zone.
%    ]
%    \begin{itemize}
%        \vspace{-0.5em}
%        \itemsep -6pt {}
%        % \item Designed and implemented simple responsive and interactive UI using only vanilla CSS. (HTML, CSS)
%%        \item Built portfolio website with {React.JS} components and vanilla {CSS}.
%%       \item Used React.JS components to create single-page application, maximizing responsiveness. (React.JS, JavaScript)
%       \item Created a modern, minimal personal portfolio using {React.JS}, React Router, and some custom CSS.
%    \end{itemize}
    
    % \customItem[
    %     title=Chess Knight Path Finder,
    %     duration=Spring 2023,
    %     keyHighlight=Used DFS graph traversal to find the most optimal path for chess knight.
    % ]
    % \begin{itemize}
    %     \vspace{-0.5em}
    %     \itemsep -6pt {}
    %     \item Given a starting and ending point, algorithm finds least number of steps to reach destination (JavaScript)
    % \end{itemize}
    
    % \customItem[
    %     title=NPM dom package...
    % ]
    % \customItem[
    %     title=Weather Web App,
    %     duration=Spring 2023,
    %     keyHighlight=Used variety of web APIs to develop weather forecast web app with dynamic UI elements
    % ]
    % \begin{itemize}
    %     \vspace{-0.5em}
    %     \itemsep -6pt {}
    %     \item Used asynchronous programming to fetch and render data from OpenWeatherMap.org (JavaScript)
    %     \item Bundled site assets using Webpack to minimize deployment size (Webpack)
    % \end{itemize}
    
%    \customItem[
%        title=\href{https://github.com/AshkanArabim/todolist}{To Do List Web App },
%        duration=Fall 2022,
%        % keyHighlight=
%    ]
%    \begin{itemize}
%        \vspace{-0.5em}
%        \itemsep -6pt {}
%        \item Developed interactive to-do list web app with local save function through vanilla {HTML{,} CSS \& JS}.
%              % \item Developed all internal logic from scratch following OOP principles % (JavaScript)
%        \item Used Chrome's local storage API for saving user data. % (JavaScript)
%        \item Followed typical {git/GitHub} version control workflow during implementation. % (git, GitHub)
%    \end{itemize}
    
%         \customItem[
%         title=\href{https://github.com/AshkanArabim/tic-tac-toe}{Tic Tac Toe Web Application },
%         duration=Fall 2022,
%%         keyHighlight=Minimal Tic-Tac-Toe game with human-human{,} human-bot{,} and bot-bot game-modes.
%         ]
%         \begin{itemize}
%             \vspace{-0.5em}
%             \itemsep -6pt {}
%             % \item Implemented smooth animations using CSS keyframes to improve site nativation. (CSS)
%%             \item Utilized CSS glow{,} shading{,} and animations for a modern and sleek look. (HTML, CSS)
%%             \item Followed OOP principles to track game progress. (JavaScript)
%%             \item Used typical git/GitHub workflow during implementation (git, GitHub)
%%             ------------------------------ OLD BULLETS, DON'T USE ^^ ---------------------------------------------
%             \item Developed a web-based tic-tac-toe application using {HTML, CSS, and JavaScript}, with a special focus on the visuals.
%             \item Used JavaScript to manage game state, handle player moves, and determine game outcomes.
%%             \item Utilized CSS glow{,} shading{,} and animations for a Neumorphist UI style.
%         \end{itemize}

%    \customItem[
%        title=\href{https://github.com/AshkanArabim/blabber}{Blabber {-} a CLI Twitter replica },
%        duration=Fall 2022,
%        % keyHighlight=Made CLI app that allows users to create an account{,} post{,} follow others{,} see a timeline{,} and delete their account
%    ]
%    \begin{itemize}
%        \vspace{-0.5em}
%        \itemsep -6pt {}
%        \item Made {Java} CLI app for users to create an account{,} post{,} follow others{,} see a timeline{,} and delete their account.
%        \item Used scanners and writers to save{,} update{,} and delete user information and posts. % (Java)
%    \end{itemize}

    % \customItem[
    %     title=Contribution to Monkeytype,
    %     duration=Spring 2021,
    %     keyHighlight=Added 3 levels of Persian tests to open source typing test website
    % ]
    % \begin{itemize}
    %     \vspace{-0.5em}
    %     \itemsep -6pt {}
    %     \item Developed script to extract and check words from large text bodies using Vajehyab.com dictionary API (Python)
    %     \item Uploaded more than 21,000 most used Persian words to site database (git, GitHub)
    % \end{itemize}
    
\end{workSection}

\begin{workSection}{Leadership \& Activities}
	% \customItem[
	% 	title=Member - UTEP ICPC Team,
	% 	duration=August 2025 - present
	% ]
	% \begin{itemize}
	% 	\vspace{-0.5em}
	% 	\itemsep -6pt {}
	% 	\item Training weekly to solve algorithmic problems for the 2025 ICPC South Central USA Regional Contest.
	% \end{itemize}

	\customItem[
		title=President \& Founder - \href{https://www.instagram.com/foss.utep/}{{FOSS Club} },
		duration=December 2023 - March 2025
	]
	\begin{itemize}
		\vspace{-0.5em}
		\itemsep -6pt {}
		%    	\item Encouraged participation in open-source projects through workshops, info sessions, competitions, and social events.
		% note: ^^ is a useless "summary". nobody cares
		\item Founded UTEP's first open-source club, growing to \textbf{264 members} \& organizing \textbf{47 technical workshops}.
		% note: 
		
		% \item Scaled UTEP's first open-source-focused student organization from \textbf{0 to 264 members in 3 semesters}. % see post-resignation stats
		% \item Organized \textbf{47 workshops, personally presenting 20}, covering topics such as Git, Linux, \& Kubernetes.
		% note: prev, but expanded into two lines
		
		% \item Led team of 6 officers to host weekly workshops on git, Linux, Vim, open-source software development, and more.
		% note: 30 is plan b. (orig. 12)
		% \item Hosted UTEP's {first open-source hackathon}, OpenHack, attended by {30+ first-time contributors}.
		% \item Hosted OpenHack, UTEP's {first open-source hackathon}, attended by \textbf{30+ first-time contributors}.
	\end{itemize}
		
	% \customItem[
	% 		title=CS Student Council,
	% 		%         keyHighlight=One of the founders{,} and current member of the Council{,} helping students voice their concerns to the administration.
	% 		keyHighlight=Empowered the CS students to voice their suggestions by providing feedback channels and hosting town halls.,
	% 		duration=Fall 2023 - Present
	% ]
	% \begin{itemize}
	% 		\vspace{-0.5em}
	% 		\itemsep -6pt {}
	% 		% \item Drafted ...
	% \end{itemize}
				
% 	\customItem[
% 			title=Treasurer - Association for Computing Machinery at UTEP,
% 			%        keyHighlight=Multiple roles{,} including publicity officer \& treasurer,
% 			% keyHighlight=Promoted side-projects and research by hosting informational and technical workshops to 40+ students each semester.,
% 			duration=August 2022 - January 2024
% 	]
% 	\begin{itemize}
% 			\vspace{-0.5em}
% 			\itemsep -6pt {}
% %        \item Planned and executed the Sun City Hackathon; a three-day competition attended by {more than 20 students} to develop novel AI-powered apps.
% 			\item Held the Sun City Hackathon; a three-day competition attended by {more than 20 students} to develop AI-powered apps.
% %%        \item Promoted side-projects and research by hosting informational and technical workshops to 40+ students each semester.
% %              % \item Recruited 50+ members through public presentations. # bad
% %              % \item Engaged with the student body to recruit more than 50 students.
% %              % \item Organized and planned of more than 6 workshops sponsored by companies such as Google.
% %              % \item Presented multiple interactive workshops attended by org members on topics such as Node.JS.
% %              %        \item Promoted research and open-source contributions by hosting info sessions and technical workshops to 40+ students each semester.
% 	\end{itemize}
\end{workSection}

\begin{workSection}{Skills}
	% note: NOBODY CARES ABOUT SKILL RATINGS!!!
	\skillItem[
		category=Coursework,
		skills=
		data structures and algorithms{,} % note: THIS ALWAYS STAYS
		operating systems{,}
		parallel computing{,} 
		% calculus i \& ii{,}
		% relational databases{,} 
		deep learning{,} 
		machine learning{,} 
		% computer vision{,} 
		% database systems{,}
		% object-oriented programming{,} 
		% numerical analysis{,} 
		% computer organization{,} 
		% computer architecture{,} % same as above. some people call it this way
		%discrete math{,}
		%digital systems design{,}
		%programming language concepts{,}
		% programming languages{,}
		%automata{,}
		% matrix algebra{,}
		% information retrieval{,}
		% computer organization{,} 
		% probability \& statistics {,}
		%introductory mechanics{,}
	] \\
	\skillItem[
		%category=Programming / Markup Languages,
		category=Languages,
		%Python: bee using heavily since Jan 2022
		%Vala: since December 2023
		%HTML/CSS: heavily since Jan 2022
		%JS: Heavily since May 2022
		%Java: since August 2022
		%Bash: heavily since June 2023
		%Haskell: used in PL, Jan - April 2024
		%LaTeX: heavily in June
		%C: heavily in Org / Hacktoberfest, August - December 2023
		skills=
		Python{,} %(adv.){,} 
		Java{,} %(adv.){,} 
		JavaScript{,} %(adv.){,} 
		TypeScript{,} %(adv.){,} 
		SQL{,} %(med.){,} 
		Go (Golang){,} 
		Dart{,} %(med.){,} 
		Vala{,} %(med.){,} 
		HTML/CSS{,} %(med.){,} 
		% UML{,} %(med.){,} 
		Bash shell{,} %(med.){,} 
		% Haskell{,} %(med.){,} 
		% \LaTeX{,} %(novice){,} 
		% C++{,} %(novice){,} 
		% C{,} %(novice){,} 
		% to be fair, I used C++ more in-depth than C. plus, it looks more dope
		% PHP{,} %(novice){,} 
	]
	\\
	\skillItem[
		category=Libraries,
		% for data science / ML:
		skills=
		%GTK(med.){,} 
		ReactJS{,} %(adv.){,} 
		PyTorch{,} %(med.){,} 
		TensorFlow{,} %(med.){,} 
		scikit-learn{,} 
		NumPy{,} %(med.){,} 
		% Pandas{,} %(med.){,} 
		Django{,} %(med.){,}
		JUnit{,} 
		% Mockito{,}
		FastAPI{,} %(med.){,} 
		% Supabase{,} %(med.){,} 
		% Firebase{,} %(med.){,} 
		Flask{,} %(med.){,} 
		Spring Boot{,} %(med.){,} % note: plan b
		% Redux{,} %(med.){,} 
		% Node.JS{,} %(med.){,} 
		% RabbitMQ{,} %(novice){,} 
		% PHP{,} %(novice){,} % I used this for the DB final project
		% MATLAB{,} %(novice){,} 
		% Vue.JS{,} %(novice){,} % plan B
		% Matplotlib{,} %(novice) 
	]
	\\
	\skillItem[
		category=Tools,
		skills=
		% Unix/Linux{,} %(adv.){,} 
		Linux{,} %(adv.){,} 
		Git{,} %(adv.){,} 
		GitHub{,} %(adv.){,} 
		%GitLab{,} %(med.){,} 
		% RDBMS{,} %(adv.){,} 
		Docker{,} %(adv.){,} 
		% Docker Compose{,} %(med.){,} 
		% Docker Swarm{,} %(med.){,} 
		% Observe{,} %(med.){,}
		AWS{,} %(novice){,}
		Google Cloud(GCP){,} %(novice){,}
		% Kubernetes{,} %(novice){,} 
		% Make{,} %(novice){,} 
		% Snowflake{,} %(novice){,}
		MongoDB(NoSQL){,} 
		PostgreSQL{,} %(med.){,} 
		MySQL{,} %(med.){,} 
		Kafka{,}
		Cursor{,}
		% GraphQL{,} 
		% New Relic{,} %(novice){,}
		% Bazel{,} %(novice){,} % plan B
		% imagemagick 
		% Conda{,} 
		% Other stuff: -----------------
		% Microsoft Office{,} 
		% Adobe Photoshop{,} 
		%Adobe Illustrator{,} 
		% nah: -----------------------
		% SSH{,} Singularity{,} Webpack {,} Jupyter,
		% DON'T INCLUDE:
		% Webpack
	]
	% \\
	%\skillItem[
	%    category=Certifications,
	%    skills=Coursera Deep Learning Specialization {-} October 2023
	%]
	% \\
	% \skillItem[
	%     category=Foreign Languages,
	%     skills=Persian (Native){,} Spanish (Fluent){,} French (Beginner)
	% ]
	% \\
	% % note: keep this lower case to save space
	% \skillItem[
	% 	category=Concepts,
	% 	skills=
	% 	containerization{,} 
	% 	unit testing{,} 
	% 	% test driven development(TDD){,} 
	% 	multi-threading{,} 
	% 	concurrency{,} 
	% 	% Integration testing{,} 
	% 	% Agile{,} 
	% 	% Scrum{,} 
	% 	AI/ML{,} 
	% 	LLMs{,} 
	% 	CI/CD{,} 
	% 	DevOps{,} 
	% 	% distributed systems{,} 
	% 	code review{,} 
	% ]
\end{workSection}

% \begin{workSection}{Publications}
% cite ACM Mobihoc paper
% \end{workSection}

\end{document}