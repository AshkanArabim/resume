%%%%%%%%%%%%%%%%%%%%%%%%%%%%%%%%%%%%%%%%%%%%%%%%%%%%%%%%
% Author: Vignesh Iyer                                 %
% MS CSE ASU                                           %
%%%%%%%%%%%%%%%%%%%%%%%%%%%%%%%%%%%%%%%%%%%%%%%%%%%%%%%%
% Modified by Ashkan Arabi

\documentclass{resume} % Use the custom resume.cls style
\usepackage{hyperref}
\RequirePackage{fontawesome}
%\hypersetup{
%	colorlinks=true,
%	linkcolor=blue,
%	urlcolor=cyan
%}

\begin{document}

\introduction[
    fullname=Ashkan Arabi,
    phone=(915) 888 - 9801,
    email=aarabimian@miners.utep.edu,
    linkedin=linkedin.com/in/ashkan-arabi,
    % github=ashkan.zone,
    github=github.com/AshkanArabim,
]

% \summary{Senior mechanical engineering student with internship experience in medical device manufacturing and product development. Project experience includes applications of software and hardware. Seeking full-time position May 2020 in medical device manufacturing and  pharmaceutical production. }

\education{
\educationItem[
university=University of Texas at El Paso{,} El Paso{,} TX,
graduation=05/26,
grade=3.95,
majorgrade=4.00,
program=Bachelor of Science in Computer Science{,} Mathematics Minor,
coursework=
Machine Learning{,} 
Object-Oriented Programming{,} 
Data Structures \& Algorithms{,} 
%Discrete Math{,}
%Operating Systems {,}
%Digital Systems Design{,}
%Programming Language Concepts{,}
%Automata{,}
Matrix Algebra %{,}
% Computer Organization{,} 
%Probability \& Statistics {,}
 %Introductory Mechanics{,}
 , 
%honors=Dean's List since Fall '22 {,} Houston Endowment Scholarship Recipient{,}
% Undergraduate Fellows{,} 
%,
]
}

\skills{
\skillItem[
%category=Programming / Markup Languages,
category=Languages,
%Python: bee using heavily since Jan 2022
%Vala: since December 2023
%HTML/CSS: heavily since Jan 2022
%JS: Heavily since May 2022
%Java: since August 2022
%Bash: heavily since June 2023
%Haskell: used in PL, Jan - April 2024
%LaTeX: heavily in June
%C: heavily in Org / Hacktoberfest, August - December 2023
skills=Python(advanced){,} Java(advanced){,} Vala(mid){,} HTML/CSS(mid){,} JavaScript(mid){,} UML(mid){,} Bash shell scripting(mid){,} Haskell(mid){,} 
LaTeX(novice){,}  C++(novice) {,} C(novice){,} SQL(novice)
% to be fair, I used C++ more in-depth than C. plus, it looks more dope
%PHP{,} 
]
\\
\skillItem[
category=Tools,
% for data science / ML:
skills=Linux(advanced){,} Git(advanced){,} GTK(mid){,}  PyTorch(mid){,} TensorFlow(mid){,} NumPy(mid){,}Pandas(mid){,}Matplotlib(novice)
%React.JS(novice) 
% nobody except web devs cares about react
% Conda{,}
% Microsoft Office,
%Docker{,} 
% Other stuff:
% skills=Adobe Photoshop{,} Adobe Illustrator{,},
% mah:
% Django {,} Microsoft Office {,} SSH{,} Singularity{,} Webpack {,} Jupyter, Node.js
% DON'T INCLUDE:
% Node.JS, Webpack
]
%\\
%\skillItem[
%    category=Certifications,
%    skills=Coursera Deep Learning Specialization {-} October 2023
%]
% \\
% \skillItem[
%     category=Foreign Languages,
%     skills=Persian (Native){,} Spanish (Fluent){,} French (Beginner)
% ]
}

%%%%%%%%%%%
% Accomplished [X] as measured by [Y], by doing [Z]
%%%%%%%%%%%

\begin{workSection}{Experience}
	
	\experienceItem[
	company=Texas Instruments,
	location=Dallas{,} TX,
	position=Information Technology Intern,
	duration= May 2024 - present,
	]
	\begin{itemize}
		\vspace{-0.5em}
		\itemsep -6pt {}
%		\item Maintained automation software for chip fabrication operations.
%		\item Developed scripts for carrier robots in AutoShell (proprietary scripting language) to use in chip fabrication.
%		\item Set up infrastructure for manufacturing automation by configuring and deploying Docker containers.
		\item Set up infrastructure for \textbf{fab transporter robots} by setting up \textbf{Docker} containers from another department.
		\item Coordinated with 10+ automation engineers and sysadmins to configure Oracle databases and internal APIs.
	\end{itemize}
	
	\experienceItem[
	company=GNOME Foundation,
	location=Remote,
	position=Open-Source Contributor,
	duration=December 2023 {-} present
	]
	\begin{itemize}
	\vspace{-0.5em}
	\itemsep -6pt {}
	%		\item Improved the GNOME Clocks user experience by fixing various reported bugs.
%	\item Contributed to the development of GNOME Clocks, an \textbf{app used by thousands of Linux users} to keep track of time.
	\item Collaborated with software engineers in development of GNOME Clocks; \textbf{used by thousands of Linux users} to track  time.
	\item Added features such as full-screen timers and timer editing. (in progress)
	\item Fixed timers not progressing during system suspend by revising timer logic. (merged)
	\item Added functionality to world clock to show country and state when two cities have the same name. (open)
	%		\item Translated the GNOME Builder IDE to Persian. % <<- they seriously don't care. only use if you have space
	%		\item Expanded the accessibility of the GNOME Builder IDE by translating it to Persian.
	\end{itemize}
	
	\experienceItem[
	company=UTEP,
	location=El Paso{,} TX,
	position=Undergraduate Research Assistant,
	duration=January 2024 - May 2024,
	]
	\begin{itemize}
		\vspace{-0.5em}
		\itemsep -6pt {}
		\item Contributed to creation of Autistic vs neurotypical speech classifier by using generative AI to synthesize training data.
%		\item Used \textbf{PyTorch} to reconstruct state-of-the-art networks proposed in papers.
		 \item Developed an \href{https://github.com/AshkanArabim/accent-change-paper-implementation}{\textbf{accent-changer model} \faExternalLink} able to convert foreign English accents to native using \textbf{HuggingFace} pretrained models through their \textbf{Python} API.
		\item Implemented a \href{https://github.com/AshkanArabim/neural-style-transfer}{\textbf{neural style-transfer model} \faExternalLink} in \textbf{PyTorch} to learn PyTorch \& explore usage for accent-changing.
%		\item Used Python's multiprocessing library to parallelize CPU-intensive experiments.
       \item Reduced evaluation script runtime \textbf{from 24+ hours to 10 minutes} by rewriting loops as higher-dimension tensor operations.
	\end{itemize}
	
%	\experienceItem[
%	company=UTEP,
%	location=El Paso{,} TX,
%	position=Undergraduate TA for Data Structures \& Algorithms,
%	duration=January 2024 - present,
%	]
%	\begin{itemize}
%		\vspace{-0.5em}
%		\itemsep -6pt {}
%		\item Helped 30+ students understand challenging concepts such as B-Trees and Dijkstra's Algorithm.
%	\end{itemize}

%    \experienceItem[
%        company=AI4ALL,
%        location=Remote,
%        position=Student Coordinator,
%        duration=August {-} December 2023
%    ]
%    \begin{itemize}
%        \vspace{-0.5em}
%        \itemsep -6pt {}
%        \item Mentored 16 students across 4 project groups in AI4ALL's Apply AI program.
%%        \item Helped students understand AI/ML concepts such as loss, backpropagation, CNNs, RNNs, reinforcement learning, etc.
%        \item Helped students understand concepts such as loss functions, optimization, neural networks, CNNs,
%%         RNNs, 
%        reinforcement learning, etc.
%    \end{itemize}
    % \experienceItem[
    %     company=UTEP,
    %     location=El Paso{,} TX,
    %     position=Tech Support Staff,
    %     duration=January 2023 {-} Present
    % ]
    % \begin{itemize}
    %     \vspace{-0.5em}
    %     \itemsep -6pt {}
    %     \item Provided top-tier hardware \& software tech support for 10+ students, staff, and faculty daily.
    %     \item Maintained and updated hardware \& software of 500+ computers through regular checkups.
%     \end{itemize}
%    \experienceItem[
%    company=UTEP,
%    location=El Paso{,} TX,
%    position=Volunteer Researcher,
%    duration=August {-} October 2023
%    ]
%    \begin{itemize}
%    \vspace{-0.5em}
%    \itemsep -6pt {}
%    \item Contributed to the development of software for training users' weaknesses in spotting Phishing emails.
%    \item Classified emails across 7 phishing attributes 
%%    (e.g. sender\_mismatch, suspicious\_subject) 
%    with 78\% accuracy by developing multi-output NLP classification model.
%%    \item Trained deep transformer model using \textbf{Keras} to classify email cues, such as a sense of urgency.
%    \item Preprocessed and visualized more than 5 datasets using \textbf{ and NLTK} to use as training material for model.
%    \end{itemize}
    \experienceItem[
    company=Temple University,
    location=Philadelphia{,} PA,
    position=Undergraduate Research Intern,
    duration=June {-} July 2023
    ]
    \begin{itemize}
        \vspace{-0.5em}
        \itemsep -6pt {}
%        \item Published
%        \href{https://dl.acm.org/doi/10.1145/3565287.3617613}{\textbf{paper} \faExternalLink} 
%        presented at ACM MobiHoc about using Wi-Fi CSI for hand gesture recognition on smartphones.
        \item Wrote
        \href{https://dl.acm.org/doi/10.1145/3565287.3617613}{\textbf{first-author publication} \faExternalLink} 
        presented at ACM MobiHoc about using Wi-Fi CSI for hand gesture recognition on phones.
        \item Developed CNN architecture to classify 5 gestures from 4 people in 6 scenarios using \textbf{Keras}. %(TensorFlow, Python, Bash)
        \item Obtained >90\% classification accuracy by using techniques such as LR Scheduling.
        \item \textbf{1st place} for the best REU site final presentation. %(TensorFlow, Keras)
%        \item Used \textbf{Bash, Android Debug Bridge, and NumPy} for data extraction and preprocessing.
              % \item 
    \end{itemize}

\end{workSection}

\begin{workSection}{Projects}
	
%	\customItem[
%		title=\href{https://github.com/AshkanArabim/pwaang-extended}{PWAANG - a board game written in Haskell \faExternalLink},
%		duration=May 2024,
%		% keyHighlight=
%	]
%	\begin{itemize}
%		\vspace{-0.5em}
%		\itemsep -6pt {}
%		%		\item Implemented a board game entirely in \textbf{Haskell}, using monads, pattern matching, and recursion.
%		\item Implemented a board game in \textbf{Haskell} (functional programming language) using monads, pattern matching, and recursion.
%	\end{itemize}
	
	\customItem[
		title=\href{https://github.com/AshkanArabim/bombshell}{Bombshell - UNIX shell written in Python \faExternalLink},
		duration=April 2024,
		% keyHighlight=
	]
	\begin{itemize}
		\vspace{-0.5em}
		\itemsep -6pt {}
		\item Added features like pipes, redirection, and background tasks by using multi-threading and semaphores, ensuring concurrency.
	\end{itemize}
	
	\customItem[
		title=\href{https://github.com/AshkanArabim/OOP-project-1}{CLI Car Dealership \faExternalLink},
		duration=April 2024,
		% keyHighlight=
	]
	\begin{itemize}
		\vspace{-0.5em}
		\itemsep -6pt {}
%		\item Wrote an Object-Oriented car dealership software in \textbf{Java}, using inheritance, polymorphism, and interfaces.
		\item Wrote an Object-Oriented car dealership software in \textbf{Java} following MVC architecture \& OOP design patterns. % with 2300+ lines of code.
		\item Used Git \& GitHub features such as pull requests, merges and branches for to work as member of a team of 3.
%		\item Used Git \& GitHub features such as pull requests, merges, branches, and rebasing for teamwork in group of 3.
%		\item Drafted overall software design in UML use-case and class diagrams.
%		\item Followed Agile methodology to design, plan, and develop three phases of feature additions.
%		\item Included operations to buy{,} restock{,} or add cars{,} monitor the revenue of each vehicle type{,} add / remove users, etc.
%		\item Included automatic
	\end{itemize}
	
	\customItem[
		title=\href{https://github.com/AshkanArabim/gtk-timer}{Linux Timer \faExternalLink},
		duration=January 2024,
		% keyHighlight=
	]
	\begin{itemize}
		\vspace{-0.5em}
		\itemsep -6pt {}
		\item Developed a \textbf{Linux} timer application using \textbf{GTK \& Vala}, using an event-driven architecture.
		\item Implemented functionalities for starting, pausing, resetting, and editing timers, using different GTK4 widgets \& signals.
	\end{itemize}
	
%	\customItem[
%		title=\href{https://github.com/AshkanArabim/pong-msp430}{Pong for MSP430 \faExternalLink},
%		duration=November 2023,
%		% keyHighlight=
%	]
%	\begin{itemize}
%		\vspace{-0.5em}
%		\itemsep -6pt {}
%		%		\item Implemented smooth graphics using interrupts to avoid graphical bottlenecks.
%		\item Designed and implemented a \textbf{C} Pong game for MSP430, with paddle movement, ball physics, and score tracking.
%%		\item Implemented interrupt-driven input handling for buttons, ensuring responsive gameplay with 30+ FPS.
%		\item Achieved 30+ FPS gameplay by using partial framebuffer updating instead of redrawing whole screen.
%		\item Integrated buzzer audio feedback for game events such as ball-wall collisions and score updates.
%	\end{itemize}
	
%   vv not including hacktoberfest 2023 because it was mostly wasted on translation projects...
	
%	\customItem[
%	title=Hacktoberfest 2023,
%	duration=October 2023,
%	% keyHighlight=
%	]
%	\begin{itemize}
%		\vspace{-0.5em}
%		\itemsep -6pt {}
%		\item Contributed to three open-source repositories through bug-fixes and translations.
%		\item Fixed subtitle cutoff bug in ASCII video player written in \textbf{C} by debugging the subtitle buffer. \href{https://github.com/aidancrowther/ASCIIPlay}{\faExternalLink}
%	\end{itemize}

	\customItem[
	title=\href{https://github.com/AshkanArabim/advent-of-code-2022}{Advent of Code 2022 - Annual Programming Challenge \faExternalLink},
	duration=August 2023,
	% keyHighlight=
	]
	\begin{itemize}
		\vspace{-0.5em}
		\itemsep -6pt {}
		\item Coded \textbf{C++} solutions to 12 of 25 challenge questions using backtracking, graph traversal algorithms, and more.
		\item Used classes, queues, vectors, and streams to efficiently calculate results based on given inputs.
	\end{itemize}
    % \customItem[
    %     title=\href{https://github.com/AshkanArabim/email-classifier}{Adaptive Phishing Email Training System \faExternalLink},
    %     duration=August 2023 - Present,
    %     keyHighlight=Research project to develop dynamic software to train users' weaknesses in spotting phishing emails.
    % ]
    % \begin{itemize}
    %     \vspace{-0.5em}
    %     \itemsep -6pt {}
    %     \item Achieved 96\% phishing email detection accuracy to classify unlabeled emails for user training. (Keras, Python)
    % \end{itemize}

    % \customItem[
    %     title=Music Genre Classification,
    %     duration=Spring 2023,
    %     keyHighlight=Developed convolutional model to classify music genres based on their mel spectrogram.
    % ]
    % \begin{itemize}
    %     \vspace{-0.5em}
    %     \itemsep -6pt {}
    %     \item Led team of 3 to meet project deadlines by dividing tasks based on skill level (Project Management)
    %     \item Designed, tested, and implemented CNN model for classification (TensorFlow, Numpy, Conda)
    % % in case you need more:
    %% original linkedin title: AI4ALL College Pathways Participant
    % %- Led team of 3 to meet project deadlines by dividing tasks based on skill level.
    %%- Developed convolutional model to classify music genres based on their Mel spectrogram.
    %% - Learned about the applications and fundamental technical concepts of AI including the types of machine learning techniques and neural networks. 
    %% - Investigated ethical implications related to data processing and AI implementation 
    %% - Collaborated with peers on a project to preditct sudents' academic performance in college based on their background.
    % \end{itemize}


    % \end{workSection}
    % \begin{workSection}{Personal Projects}

    % in case you change your mind: ----------------
    % \experienceItem[
    % company=Arizona State University,
    % location=Tempe{,} AZ,
    % position=Tutor (10 hours/week),
    % duration=Aug 2018 – May 2019
    % ]
    % \begin{itemize}
    %     \vspace{-0.5em}
    %     \itemsep -6pt {}
    %     \item Tutored 10-15 undergraduate engineering students per week in MATLAB programming and math coursework
    % \end{itemize}
    % ---------------------------------------------
    
%    \customItem[
%        title=\href{https://github.com/AshkanArabim/cybertweet-topics/tree/main}{Tweet topic detection \faExternalLink},
%        duration=August 2023,
%        % keyHighlight=
%    ]
%    \begin{itemize}
%        \vspace{-0.5em}
%        \itemsep -6pt {}
%        \item Tuned \textbf{Keras} model to detect one of 8 cyber-security topics mentioned in Tweets.
%        \item Utilized Twitter pre-trained word embeddings and Bidirectional GRU to achieve 93\% accuracy.
%    \end{itemize}
    
    % \customItem[
    %     title=Tweet Sentiment Classification,
    %     duration=Summer 2023,
    %     keyHighlight=Built almost perfect model to classify positive and negative Tweets
    % ]
    % \begin{itemize}
    %     \vspace{-0.5em}
    %     \itemsep -6pt {}
    %     \item Achieved 99\% accuracy using standard NLP preprocessing and along with Embedding and LSTM layers. (Keras)
    % \end{itemize}
    
%    \customItem[
%        title=\href{https://ashkan.zone/}{Personal Portfolio \faExternalLink},
%        duration=Spring 2023,
%        % keyHighlight=Built complete portfolio website with React.JS and vanilla CSS. See https://ashkan.zone.
%    ]
%    \begin{itemize}
%        \vspace{-0.5em}
%        \itemsep -6pt {}
%        % \item Designed and implemented simple responsive and interactive UI using only vanilla CSS. (HTML, CSS)
%        \item Built portfolio website with \textbf{React.JS} components and custom vanilla \textbf{CSS}.
%              % \item Used React.JS components to create single-page application, maximizing responsiveness. (React.JS, JavaScript)
%    \end{itemize}
    
    % \customItem[
    %     title=Chess Knight Path Finder,
    %     duration=Spring 2023,
    %     keyHighlight=Used DFS graph traversal to find the most optimal path for chess knight.
    % ]
    % \begin{itemize}
    %     \vspace{-0.5em}
    %     \itemsep -6pt {}
    %     \item Given a starting and ending point, algorithm finds least number of steps to reach destination (JavaScript)
    % \end{itemize}
    
    % \customItem[
    %     title=NPM dom package...
    % ]
    % \customItem[
    %     title=Weather Web App,
    %     duration=Spring 2023,
    %     keyHighlight=Used variety of web APIs to develop weather forecast web app with dynamic UI elements
    % ]
    % \begin{itemize}
    %     \vspace{-0.5em}
    %     \itemsep -6pt {}
    %     \item Used asynchronous programming to fetch and render data from OpenWeatherMap.org (JavaScript)
    %     \item Bundled site assets using Webpack to minimize deployment size (Webpack)
    % \end{itemize}
    
%    \customItem[
%        title=\href{https://github.com/AshkanArabim/todolist}{To Do List Web App \faExternalLink},
%        duration=Fall 2022,
%        % keyHighlight=
%    ]
%    \begin{itemize}
%        \vspace{-0.5em}
%        \itemsep -6pt {}
%        \item Developed interactive to-do list web app with local save function through vanilla \textbf{HTML{,} CSS \& JS}.
%              % \item Developed all internal logic from scratch following OOP principles % (JavaScript)
%        \item Used Chrome's local storage API for saving user data. % (JavaScript)
%        \item Followed typical \textbf{git/GitHub} version control workflow during implementation. % (git, GitHub)
%    \end{itemize}
    
         \customItem[
         title=\href{https://github.com/AshkanArabim/tic-tac-toe}{Tic Tac Toe Web Application \faExternalLink},
         duration=Fall 2022,
%         keyHighlight=Minimal Tic-Tac-Toe game with human-human{,} human-bot{,} and bot-bot game-modes.
         ]
         \begin{itemize}
             \vspace{-0.5em}
             \itemsep -6pt {}
             % \item Implemented smooth animations using CSS keyframes to improve site nativation. (CSS)
%             \item Utilized CSS glow{,} shading{,} and animations for a modern and sleek look. (HTML, CSS)
%             \item Followed OOP principles to track game progress. (JavaScript)
%             \item Used typical git/GitHub workflow during implementation (git, GitHub)
%             ------------------------------ OLD BULLETS, DON'T USE ^^ ---------------------------------------------
             \item Developed a web-based tic-tac-toe application using \textbf{HTML, CSS, and JavaScript}, with a special focus on the visuals.
             \item Used JavaScript to manage game state, handle player moves, and determine game outcomes.
%             \item Utilized CSS glow{,} shading{,} and animations for a Neumorphist UI style.
         \end{itemize}

%    \customItem[
%        title=\href{https://github.com/AshkanArabim/blabber}{Blabber {-} a CLI Twitter replica \faExternalLink},
%        duration=Fall 2022,
%        % keyHighlight=Made CLI app that allows users to create an account{,} post{,} follow others{,} see a timeline{,} and delete their account
%    ]
%    \begin{itemize}
%        \vspace{-0.5em}
%        \itemsep -6pt {}
%        \item Made \textbf{Java} CLI app for users to create an account{,} post{,} follow others{,} see a timeline{,} and delete their account.
%        \item Used scanners and writers to save{,} update{,} and delete user information and posts. % (Java)
%    \end{itemize}

    % \customItem[
    %     title=Contribution to Monkeytype,
    %     duration=Spring 2021,
    %     keyHighlight=Added 3 levels of Persian tests to open source typing test website
    % ]
    % \begin{itemize}
    %     \vspace{-0.5em}
    %     \itemsep -6pt {}
    %     \item Developed script to extract and check words from large text bodies using Vajehyab.com dictionary API (Python)
    %     \item Uploaded more than 21,000 most used Persian words to site database (git, GitHub)
    % \end{itemize}
    
\end{workSection}

\begin{workSection}{Leadership}
    % \customItem[
    %     title=CS Student Council,
    %     %         keyHighlight=One of the founders{,} and current member of the Council{,} helping students voice their concerns to the administration.
    %     keyHighlight=Empowered the CS students to voice their suggestions by providing feedback channels and hosting town halls.,
    %     duration=Fall 2023 - Present
    % ]
    % \begin{itemize}
    %     \vspace{-0.5em}
    %     \itemsep -6pt {}
    %     % \item Drafted ...
    % \end{itemize}
    
    \customItem[
    	title=President \& Founder - \href{https://www.instagram.com/foss.utep/}{\textbf{Free and Open Source Software Club at UTEP} \faExternalLink},
    	duration=December 2023 - Present
    ]
    \begin{itemize}
    	\vspace{-0.5em}
    	\itemsep -6pt {}
%    	\item Encouraged participation in open-source projects through workshops, info sessions, competitions, and social events.
    	\item Recruited team of 6 officers to host weekly workshops on git, Linux, Vim, open-source software development, and more.
		\item Hosted UTEP's \textbf{first open-source hackathon}, OpenHack, attended by more than \textbf{12 first-time contributors}.
    \end{itemize}

    \customItem[
        title=Treasurer - Association for Computing Machinery at UTEP,
        %        keyHighlight=Multiple roles{,} including publicity officer \& treasurer,
        % keyHighlight=Promoted side-projects and research by hosting informational and technical workshops to 40+ students each semester.,
        duration=August 2022 - January 2024
    ]
    \begin{itemize}
        \vspace{-0.5em}
        \itemsep -6pt {}
%        \item Planned and executed the Sun City Hackathon; a three-day competition attended by \textbf{more than 20 students} to develop novel AI-powered apps.
        \item Held the Sun City Hackathon; a three-day competition attended by \textbf{more than 20 students} to develop AI-powered apps.
%%        \item Promoted side-projects and research by hosting informational and technical workshops to 40+ students each semester.
%              % \item Recruited 50+ members through public presentations. # bad
%              % \item Engaged with the student body to recruit more than 50 students.
%              % \item Organized and planned of more than 6 workshops sponsored by companies such as Google.
%              % \item Presented multiple interactive workshops attended by org members on topics such as Node.JS.
%              %        \item Promoted research and open-source contributions by hosting info sessions and technical workshops to 40+ students each semester.
    \end{itemize}
\end{workSection}

% \begin{workSection}{Publications}
% cite ACM Mobihoc paper
% \end{workSection}

\end{document}