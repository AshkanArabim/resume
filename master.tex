%%%%%%%%%%%%%%%%%%%%%%%%%%%%%%%%%%%%%%%%%%%%%%%%%%%%%%%%
% Author: Vignesh Iyer                                 %
% MS CSE ASU                                           %
%%%%%%%%%%%%%%%%%%%%%%%%%%%%%%%%%%%%%%%%%%%%%%%%%%%%%%%%
% Modified by Ashkan Arabi

\documentclass{resume} % Use the custom resume.cls style
\usepackage{hyperref}
\RequirePackage{fontawesome}
%\hypersetup{
%	colorlinks=true,
%	linkcolor=blue,
%	urlcolor=cyan
%}

\begin{document}

\introduction[
    fullname=Ashkan Arabi,
    phone=(915) 888 - 9801,
    email=aarabimian@miners.utep.edu,
    linkedin=linkedin.com/in/ashkan-arabi,
    % github=ashkan.zone,
    github=github.com/AshkanArabim,
    % github=\faGithub,
]

% \summary{Senior mechanical engineering student with internship experience in medical device manufacturing and product development. Project experience includes applications of software and hardware. Seeking full-time position May 2020 in medical device manufacturing and  pharmaceutical production. }

\education{
% \educationItem[
% university=The University of Texas at El Paso,
% graduation=December 2025,
% % grade=3.93, % after spring 25 semester
% majorgrade=4.00,
% program=Bachelor's of Science in Computer Science, %{,} Minor in Mathematics ,
% % note: bloomberg in spring 25 said coursework is always required
% coursework=
% Parallel Computing{,} 
% % Calculus I \& II{,}
% % Relational Databases{,} 
% % Machine Learning{,} 
% % Computer Vision{,} 
% Data Structures and Algorithms{,} % note: THIS ALWAYS STAYS
% Databases{,}
% % Object-Oriented Design{,} 
% % Numerical Analysis{,} 
% % Computer Organization{,} 
% % Computer Architecture{,} % same as above. some people call it this way
% %Discrete Math{,}
% Operating Systems%{,}
% %Digital Systems Design{,}
% %Programming Language Concepts{,}
% % Programming Languages{,}
% %Automata{,}
% % Matrix Algebra %{,}
% % Computer Organization{,} 
% % Probability \& Statistics {,}
% %Introductory Mechanics{,}
% , 
% %honors=Dean's List since Fall '22 - present {,} Houston Endowment Scholarship Recipient{,} CS Engagement and Leadership Award (spring 25) {,} 
% % Undergraduate Fellows{,} 
% % University Interscholastic League, 3rd place in computer science
% %,
% ]

% \textbf{Master of Science in Artificial Intelligence} \hfill {December 2026} \\
\textbf{Bachelor of Science in Computer Science} \hfill {December 2025} \\
The University of Texas at El Paso \hfill Major GPA: 4.00 \\
Relevant coursework: Parallel Computing{,} Data Structures and Algorithms{,} Databases{,} Operating Systems
}

%%%%%%%%%%%
% Accomplished [X] as measured by [Y], by doing [Z]
%%%%%%%%%%%

\begin{workSection}{Experience}
	
	\experienceItem[
	company=Capital One,
	location=McLean{,} VA,
	position=Software Engineering Intern,
	duration= June 2025 - present,
	]
	\begin{itemize}
		\vspace{-0.5em}
		\itemsep -6pt {}
		% note: plan b stuff:
		% \item Created data pipeline for fraud detection during payment processing using \textbf{Java}, \textbf{Spring Boot}, \& \textbf{Hibernate}.
		% \item Created fraud detection data pipeline using \textbf{Java}, \textbf{Spring Boot}, \& \textbf{Hibernate}, contributing to projected \textbf{23\% reduction in fraud}.
		% note: project 1:
		\item Built 
		% root-cause-analysis dashboard 
		alerting pipeline 
		to detect card service failures \& notify SREs, reducing incident-response time by projected \textbf{61\%}.
		% \item Leveraged internal tools to monitor \textbf{Snowflake} data and spot performance anomalies.
		% note: project 2:
		% gotta update vv if I end up automating the process
		\item Migrated monitoring dashboards from \textbf{New Relic} and \textbf{Splunk} to \textbf{Observe}, solving observability fragmentation.
		% note: project 3:
		% \item Streamlined developer experience \textbf{Java} library tracking data lineage across card authorization system.
		% note: technically, consumer lineage is more broad. "consumer" just means other teams can use the plugin.
		% note: you could also add Mockito here.
		% \item Extended \textbf{Java} data lineage plugin to support consumer spending data, improving audit-readiness.
		\item Extended \textbf{Java} data lineage plugin to support consumer spending data though test-driven development with \textbf{JUnit}.
	\end{itemize}
	
	\experienceItem[
	company=UTEP,
	location=El Paso{,} TX,
	position=Undergraduate Research Assistant,
	duration= January 2025 - May 2025,
	]
	\begin{itemize}
		\vspace{-0.5em}
		\itemsep -6pt {}
		% 8+ might be considered "plan B"
		% \item Provisioned \textbf{Docker Swarm} system to stream hardware usage information from \textbf{8+} compute nodes into one dashboard.
		\item Provisioned \textbf{Docker Swarm} 
		distributed 
		system to monitor hardware information of 8-node GPU cluster.
		% to stream hardware information 
		% over the network
		% from \textbf{8+} compute nodes into one dashboard.
		\item Created and tested \textbf{Docker and Apptainer} containers for Nvidia DGX GPU clusters to facilitate reproducible HPC research.
	\end{itemize}
	
	\experienceItem[
	company=Texas Instruments,
	location=Dallas{,} TX,
	position=Information Technology Intern,
	duration= May 2024 - August 2024,
	]
	\begin{itemize}
		\vspace{-0.5em}
		\itemsep -6pt {}
		% \item Maintained automation software for chip fabrication operations.
		% \item Developed scripts for carrier robots in AutoShell (proprietary scripting language) to use in chip fabrication.
		% \item Set up infrastructure for manufacturing automation by configuring and deploying \textbf{Docker} containers.
		\item Set up \textbf{robotic automation} infrastructure by deploying \textbf{Docker} containers on \textbf{Linux} servers, reducing costs by up to \textbf{50\%}. % 50% is plan b
		% \item Set up infrastructure for \textbf{fab transporter robots} by configuring \textbf{Docker} containers on \textbf{Linux} servers.
		% \item Ensured system reliability through unit and integration testing with Insomnia.
		% \item Ensured system reliability through unit and integration testing with Insomnia before final deployment.
		% \item Coordinated with \textbf{10+} sysadmins \& stakeholders to configure Oracle \textbf{SQL} databases and internal \textbf{APIs}.
		\item Coordinated with \textbf{10+} stakeholders to configure \textbf{Oracle SQL} databases, \textbf{REST APIs}, and documentation.
		% \item Created Oracle SQL tables and views feeding 
	\end{itemize}
	
	\experienceItem[
	company=GNOME Foundation,
	location=Remote,
	position=Open-Source Contributor,
	duration=December 2023 {-} June 2024
	]
	\begin{itemize}
		\vspace{-0.5em}
		\itemsep -6pt {}
		% note: commits and their status
		% - src: https://gitlab.gnome.org/GNOME/gnome-clocks/-/merge_requests/?sort=created_date&state=all&author_username=AshkanArabim&first_page_size=20
		%   - open: https://gitlab.gnome.org/GNOME/gnome-clocks/-/merge_requests/308
		%   - merged: https://gitlab.gnome.org/GNOME/gnome-clocks/-/merge_requests/309
		%   - open: https://gitlab.gnome.org/GNOME/gnome-clocks/-/merge_requests/314

		%		\item Improved the GNOME Clocks user experience by fixing various reported bugs.
		%	\item Contributed to the development of GNOME Clocks, an \textbf{app used by thousands of Linux users} to keep track of time.
		% \item (CHANGE THIS) learned Vala and GTK *on my own!!!!*. only using online docs and code examples
		% \item Contributed to GNOME Clocks; \textbf{used by thousands of Linux users} to track  time. % note: bloomberg said this is shit.
		\item Contributed to GNOME Clocks to refine timer UX; \textbf{used by 40+ million} to track  time. % number of ubuntu users, pulled from wikipedia
		\item Added 
		% features such as 
		full-screen timers and timer editing using \textbf{Vala} bindings and \textbf{GTK} components \href{https://youtu.be/fDzYWusOLow}{(\underline{video demo \faExternalLink})}. % (in progress)
		% \item Solved timers not progressing during system suspend by revising timer logic. % (merged)
		% \item Learned completely new language and framework \textbf{in one week} before starting development.
		% \item Authored functionality for world clock to show country and state when two cities have the same name. % (open)
		%		\item Translated the GNOME Builder IDE to Persian. % <<- they seriously don't care. only use if you have space
		%		\item Expanded the accessibility of the GNOME Builder IDE by translating it to Persian.
		% \item Merged \textbf{2} pull requests that passed the CI/CD pipeline and were deployed to production. % 
		% \item Merged \textbf{2} pull requests that were deployed to production. % 
	\end{itemize}
	
	\experienceItem[
	company=UTEP,
	location=El Paso{,} TX,
	position=Undergraduate Research Assistant,
	duration=January 2024 - May 2024,
	]
	\begin{itemize}
		\vspace{-0.5em}
		\itemsep -6pt {}
		\item Contributed to creation of Autistic vs neurotypical speech classifier by
		% using generative AI to 
		synthesizing training data.
%		\item Used \textbf{PyTorch} to reconstruct state-of-the-art networks proposed in papers.
		%  \item Developed an \href{https://github.com/AshkanArabim/accent-change-paper-implementation}{\textbf{accent-changer model} \faExternalLink} able to convert foreign English accents to native using HuggingFace pretrained AI/ML models through their \textbf API.
		% \item Developed \href{https://github.com/AshkanArabim/accent-change-paper-implementation}{accent-changer model \faExternalLink} to convert foreign English accents to native using HuggingFace pretrained AI/ML models.
		\item Built \href{https://github.com/AshkanArabim/accent-change-paper-implementation}{\underline{accent-changer model \faExternalLink}} model using \textbf{HuggingFace} pretrained AI/ML to change foreign English accents to native.
		% \item Implemented a \href{https://github.com/AshkanArabim/neural-style-transfer}{\textbf{neural style-transfer model} \faExternalLink} in \textbf{PyTorch} to learn PyTorch \& explore usage for accent-changing.
		\item Implemented \href{https://github.com/AshkanArabim/neural-style-transfer}{\underline{neural style-transfer model \faExternalLink}} in \textbf{PyTorch} to learn PyTorch \& explore usage for accent-changing.
%		\item Used Python's multiprocessing library to parallelize CPU-intensive experiments.
		% \item Reduced evaluation script runtime 
		% 	%\textbf{from 24+ hrs to 10 mins} 
		% 	\textbf{by 14400\%} 
		% 	by rewriting loops as higher-dimension tensor operations
		\item Increased evaluation script performance by
			%\textbf{from 24+ hrs to 10 mins} 
			\textbf{144x} 
			reimplementing KNN in PyTorch to utilize GPU.
	\end{itemize}
	
%	\experienceItem[
%	company=UTEP,
%	location=El Paso{,} TX,
%	position=Undergraduate TA for Data Structures \& Algorithms,
%	duration=January 2024 - present,
%	]
%	\begin{itemize}
%		\vspace{-0.5em}
%		\itemsep -6pt {}
%		\item Helped 30+ students understand challenging concepts such as B-Trees and Dijkstra's Algorithm.
%	\end{itemize}

%    \experienceItem[
	%        company=AI4ALL,
%        location=Remote,
%        position=Student Coordinator,
%        duration=August {-} December 2023
%    ]
%    \begin{itemize}
%        \vspace{-0.5em}
%        \itemsep -6pt {}
%        \item Mentored 16 students across 4 project groups in AI4ALL's Apply AI program.
%%        \item Helped students understand AI/ML concepts such as loss, backpropagation, CNNs, RNNs, reinforcement learning, etc.
%        \item Helped students understand concepts such as loss functions, optimization, neural networks, CNNs,
%%         RNNs, 
%        reinforcement learning, etc.
%    \end{itemize}
    % \experienceItem[
    %     company=UTEP,
    %     location=El Paso{,} TX,
    %     position=Tech Support Staff,
    %     duration=January 2023 {-} Present
    % ]
    % \begin{itemize}
    %     \vspace{-0.5em}
    %     \itemsep -6pt {}
    %     \item Provided top-tier hardware \& software tech support for 10+ students, staff, and faculty daily.
    %     \item Maintained and updated hardware \& software of 500+ computers through regular checkups.
%     \end{itemize}

	% \experienceItem[
	% company=UTEP,
	% location=El Paso{,} TX,
	% position=Volunteer Researcher,
	% duration=August 2023 {-} October 2023
	% ]
	% \begin{itemize}
	% 	\vspace{-0.5em}
	% 	\itemsep -6pt {}
	% 	\item Contributed to the development of software for training users' weaknesses in spotting Phishing emails.
	% 	% \item Classified emails across 7 phishing attributes 
	% 	% % (e.g. sender\_mismatch, suspicious\_subject) 
	% 	% with \textbf{78\%} accuracy by developing multi-output NLP classification model.
	% 	\item Classified emails across \textbf{7} phishing attributes with \textbf{78\%} accuracy by designing \textbf{TensorFlow} NLP machine learning model.
	% 	% \item Trained deep transformer model using \textbf{Keras} to classify email cues, such as a sense of urgency.
	% 	\item Preprocessed and visualized more than 5 datasets using \textbf{Python} and \textbf{NLTK} to use as training material for model.
	% \end{itemize}

    \experienceItem[
    company=Temple University,
    location=Philadelphia{,} PA,
    position=Undergraduate Research Intern,
    duration=June 2023 {-} July 2023
    ]
    \begin{itemize}
        \vspace{-0.5em}
        \itemsep -6pt {}
%        \item Published
%        \href{https://dl.acm.org/doi/10.1145/3565287.3617613}{\textbf{paper} \faExternalLink} 
%        presented at ACM MobiHoc about using Wi-Fi CSI for hand gesture recognition on smartphones.
        \item Wrote
        \href{https://dl.acm.org/doi/10.1145/3565287.3617613}{\underline{first-author publication \faExternalLink}}{,} 
        \textbf{accepted into ACM MobiHoc '23}{,}
				about using Wi-Fi 
				% CSI 
				for hand gesture recognition on phones.
        % \item Developed CNN architecture to classify 5 gestures from 4 people in 6 scenarios using \textbf{Keras}. %(TensorFlow, Python, Bash)
        \item Developed CNN architecture to classify 5 gestures from 4 people in 6 scenarios using \textbf{TensorFlow} with \textbf{90+\% accuracy}. %(TensorFlow, Python, Bash)
%        \item Obtained >90\% classification accuracy by using techniques such as LR Scheduling.
        % \item \textbf{1st place} for the best REU site final presentation. %(TensorFlow, Keras)
%        \item Used \textbf{Bash, Android Debug Bridge, and NumPy} for data extraction and preprocessing.
              % \item 
        \item Selected to receive the Student Travel Grant Award of \textbf{\$1,200} out of hundreds of researchers to cover travel expenses.
				% note: src: email titled "RMBL REU Award Letter - Arabi". They gave the grant to almost everyone who had done an REU though.
    \end{itemize}

\end{workSection}

\begin{workSection}{Projects}
	
	\customItem[
	title=\href{https://github.com/AshkanArabim/hackerhunt}{hackerhunt.tech \faExternalLink},
	duration=Solo | January 2025 - present,
	]
	\begin{itemize}
		\vspace{-0.5em}
		\itemsep -6pt {}
		\item Democratized collaboration by making a platform for UTEP students to find collaborators for technical sideprojects. 
		\item Used \textbf{Django + ChakraUI + Redux + React-Router + TypeScript} to create fast, state-of-the-art fullstack webapp.
		% \item Used \textbf{Django REST Framework} (Service-Oriented Architecture) \& \textbf{Django ORM} to create MVP webapp.
		% \item Incorporated email verification, password resetting, and transactional notifications through \textbf{AWS SES}.
		\item Minimized operating costs by self-hosting HTTPS website using \textbf{Nginx}, \textbf{Docker Compose}, and manual DNS setup.
		% \item token-based authentication????
		\item Raised \textbf{\$1000 in pre-seed funding} for first round of market validation advertisements, helping to get \textbf{30+ beta testers}. % 30 users is plan b
		\item Created Bash \textbf{CI/CD} script to immediately deploy new releases.
		% \item TODO: add hard numbers after making it public
	\end{itemize}
	
	\customItem[
	title=Athlytix,
	% the repo is private rn: https://github.com/AshkanArabim/athlytix
	duration=Team of 4 | April 2025 - May 2025,
	]
	\begin{itemize}
		\vspace{-0.5em}
		\itemsep -6pt {}
		\item Created platform to connect MMA fighters seeking improvement with gyms/orgs seeking talent, using \textbf{Supabase} and \textbf{React}.
		% \item Enabled scalable extraction of statistics from training \& fighting videos by using \textbf{Deno serverless functions}.
		\item Won the \textbf{Innovator Award} at STTE foundation's AI hackathon.
	\end{itemize}

	% \customItem[
	% title=\href{https://github.com/chesterCaii/back-logz/}{Backlogz \faExternalLink}, % HackWesTX
	% duration=Team of 4 | April 2025,
	% ]
	% \begin{itemize}
	% 	\vspace{-0.5em}
	% 	\itemsep -6pt {}
	% 	\item Boosted personal productivity by creating an app to generate 90-second podcasts based on users' backlog of interesting topics.
	% 	\item Shipped app for \textbf{Android, iOS, and web} by using \textbf{React Native} and \textbf{Expo-Router} cross-platform frameworks.
	% 	\item Won \textbf{3rd place} in SFHacks 2025's startup track.
	% \end{itemize}
	
	% \customItem[
	% % source code in class files from spring 25
	% title=Video Stabilizer,
	% duration=Solo | April 2025,
	% ]
	% \begin{itemize}
	% 	\vspace{-0.5em}
	% 	\itemsep -6pt {}
	% 	\item Created a video stabilizer using \textbf{OpenCV}, ORB feature matching, and \textbf{RANSAC} to reduce camera shake in real-time footage.
	% \end{itemize}
	
	% \customItem[
	% title=Astrophotography Reconstruction,
	% duration=February 2025,
	% ]
	% \begin{itemize}
	% 	\vspace{-0.5em}
	% 	\itemsep -6pt {}
	% 	\item Reconstructed high-quality astrophotography from 1000 noisy frames using \textbf{lucky imaging}, edge-based sharpness ranking, and intensity correction in \textbf{NumPy/SciPy}.
	% \end{itemize}
	
	% \customItem[
	% title=Virtual Ad,
	% duration=February 2025,
	% ]
	% \begin{itemize}
	% 	\vspace{-0.5em}
	% 	\itemsep -6pt {}
	% 	\item Built a virtual ad placement system for sports broadcasting using \textbf{OpenCV}, \textbf{NumPy}, and \textbf{SKImage}.
	% \end{itemize}

	% \customItem[
	% title=PanoCam,
	% duration=February 2025,
	% ]
	% \begin{itemize}
	% 	\vspace{-0.5em}
	% 	\itemsep -6pt {}
	% 	\item Created an image stitching pipeline using \textbf{OpenCV and SKImage}, capable of combining many small images into one large image.
	% \end{itemize}

	%%%%%%%%%%
	% note: there was a bloomberg tech lab project that used Redis somewhere,
	% but I'm not including it because the rest was just React, Flask, etc.
	% really boring tech, and we didn't even finish the damn project.
	%%%%%%%%%%

	% \customItem[
	% title=PALADIN,
	% duration=February 2025 - April 2025,
	% ]
	% \begin{itemize}
	% 	\vspace{-0.5em}
	% 	\itemsep -6pt {}
	%   TODO: add the "problem solving for ai" project if you feel like it. I personally winged it so idk if it should be here.
	% \end{itemize}

	% \customItem[
	% title=\href{https://github.com/AshkanArabim/flirtify}{Flirtify \faExternalLink},
	% duration=Solo | December 2024 - January 2025,
	% ]
	% \begin{itemize}
	% 	\vspace{-0.5em}
	% 	\itemsep -6pt {}
	% 	% \item Created an Android chat app that uses Google's \textbf{Gemini API} to add a flirting tone to all sent messages.
	% 	\item Created an Android chat app that uses \textbf{Google Cloud}'s Gemini API to add a flirting tone to all sent messages.
	% 	\item Used \textbf{Flutter (Dart)} to create the mobile UI and \textbf{Firebase} to enable instant communication.
	% \end{itemize}

	% title=\href{https://github.com/AshkanArabim/news-briefer}{News Bridge - BorderHack 2024 \faExternalLink},
	% duration=Team of 4 | September 2024,
	% original title ^^
	\customItem[
		title=\href{https://github.com/AshkanArabim/newsbridge}{Newsbridge \faExternalLink},
		duration=Team of 4 | September 2024 - December 2024,
	]
	\begin{itemize}
		\vspace{-0.5em}
		\itemsep -6pt {}
		% \item Increased access to diverse news by creating app for daily briefings from RSS feeds, regardless of source language.
		% \item Increased access to diverse news by creating an AI RSS news translator \& orator with \textbf{55 GitHub stars}.
		\item Increased access to diverse news by creating an AI RSS news translator \& orator with \textbf{55 GitHub stars}.
		\item Implemented a microservice architecture using \textbf{Docker Compose}, for future load-balancing \& increased capacity.
		% \item Enabled immediate playback using \textbf{Python async programming} to generate and play one sentence at a time.
%		\item Developed app that takes RSS news feeds (from any source in any language), translates \& summarizes top stories, and plays the summary back like a news briefing podcast.
		% \item Used \textbf{Flask} with \textbf{PostgreSQL} to fetch users' news sources and pass them to LLM and TTS models for summarization.
		\item Used \textbf{FastAPI} with \textbf{PostgreSQL} to fetch users' news sources and pass them to \textbf{LLM and TTS models} for summarization.
		% \item Used \textbf{Flask} with \textbf{PostgreSQL} hosted on \textbf{Google Cloud} to store users' native language \& news sources.
		% \item Loaded users' customized UI using \textbf{Redux.JS} and \textbf{React.JS}.
		\item Programmed interactive web UI using \textbf{Redux.JS} and \textbf{React.JS}.
		% (not including other parts of project cuz I didn't make them)
		% \item Developed and tested \textbf{REST API interfaces} for team members to use in the front-end.
	\end{itemize}
	
	% \customItem[
	% title=\href{https://devpost.com/software/vocowbulary-courses}{Vocabulary Courses \faExternalLink}, % HackWesTX
	% duration=Team of 4 | September 2024,
	% ]
	% \begin{itemize}
	% 	\vspace{-0.5em}
	% 	\itemsep -6pt {}
	% 	\item Built full-stack web-app to help non-native speakers practice their pronunciation using spaced repetition. % (team of 4, 24 hours)
	% 	\item Conceived custom scheduling algorithm on \textbf{Node.JS} and \textbf{MongoDB} (noSQL) back-end to fetch the next practice word.
	% 	% \item Developed and tested \textbf{REST API interfaces} for team members to use in the front-end. % note: bloomberg said this sounds redundant
	% \end{itemize}
	
%	\customItem[
%			title=\href{https://github.com/AshkanArabim/forced-flirtation}{Forced Flirtation (in progress) - full-stack chat app with a twist \faExternalLink},
%			duration=August - September 2024,
%	]
%	\begin{itemize}
%			\vspace{-0.5em}
%			\itemsep -6pt {}
%			\item Created a chat app that uses OpenAI's ChatGPT API to add a flirting tone to all messages users send on the platform.
%			\item Used a \textbf{Node.JS} backend, paired with \textbf{MongoDB} and \textbf{Redux} to enable instant communication.
%	\end{itemize}
	
	% \customItem[
	% 	title=\href{https://blog.ashkan.zone/}{Personal Blog \faExternalLink},
	% 	duration=Solo | August 2024,
	% ]
	% \begin{itemize}
	% 	\vspace{-0.5em}
	% 	\itemsep -6pt {}
	% 	% \item Created a blog to share useful programming and life tips.
	% 	% \item Used \textbf{React.JS}, \textbf{TypeScript}, and \textbf{Tailwind CSS} to quickly bring my design to life, and \textbf{Gatsby} to add Markdown support.
	% 	% \item Used \textbf{React.JS}, \textbf{TypeScript}, and \textbf{TailwindCSS} to develop blog UI.
	% 	% \item Added Markdown support with \textbf{Gatsby}, utilizing \textbf{GraphQL} to fetch blog entries.
	% 	\item Created personal blog with \textbf{React.JS} \& \textbf{TypeScript}, and used \textbf{GraphQL} to fetch blog entries. % tailored for Meta, who doesn't care about abstract frameworks
	% \end{itemize}
	
	% \customItem[
	% 	title=\href{https://github.com/AshkanArabim/pwaang-extended}{PWAANG - board game written in Haskell \faExternalLink},
	% 	duration=Solo | May 2024,
	% 	% keyHighlight=
	% ]
	% \begin{itemize}
	% 	\vspace{-0.5em}
	% 	\itemsep -6pt {}
	% 	%		\item Implemented a board game entirely in \textbf{Haskell}, using monads, pattern matching, and recursion.
	% 	% \item Implemented a board game in \textbf{Haskell} (functional programming language) using monads, pattern matching, and recursion.
	% 	\item Implemented a full-fledged board game in \textbf{Haskell}, pushing myself to learn functional programming.
	% \end{itemize}

	% \customItem[
	% 	title=Bloomberg Tech Lab on Campus,
	% 	duration=Team of 2 | April 2024,
	% ]
	% \begin{itemize}
	% 	\vspace{-0.5em}
	% 	\itemsep -6pt {}
	% 	\item \textbf{One of 40} students selected to collaborate with Bloomberg engineers 
	% 	% in a small group setting 
	% 	to build stock data processing application. % (official)
	% 	% \item Utilized \textbf{Python} to design and implement a robust message queue system using \textbf{RabbitMQ}, enhancing real-time data processing and communication between producer and consumer components. % (official)
	% 	\item Designed and implemented a queued messaging system using \textbf{RabbitMQ} for real-time data stream processing.
	% 	% \item Designed and implemented a message passing system using \textbf{RabbitMQ} for real-time data stream processing.
	% 	% \item Developed a deeper understanding of core Python/CS concepts (Classes, Inheritance, OOP), as well as financial domain knowledge (Tickers, Industry Sectors). % (official)
	% \end{itemize}
	
%	\customItem[
%		title=\href{https://github.com/AshkanArabim/bombshell}{Bombshell - Unix shell written in Python \faExternalLink},
%		duration=April 2024,
%		% keyHighlight=
%	]
%	\begin{itemize}
%		\vspace{-0.5em}
%		\itemsep -6pt {}
%		\item Added features like pipes, redirection, and background tasks by using multi-threading and semaphores, ensuring concurrency.
%	\end{itemize}
	
	% \customItem[
	% 	title=\href{https://github.com/AshkanArabim/os-file-transfer}{TCP/IP file-transfer \faExternalLink},
	% 	duration=Solo | March 2024,
	% 	% keyHighlight=
	% ]
	% \begin{itemize}
	% 	\vspace{-0.5em}
	% 	\itemsep -6pt {}
	% 	\item Built custom client-server system to stream files over basic \textbf{TCP sockets} from \textbf{Python}'s "socket" networking library.
	% 	% \item Enabled parallel transfers using forked child processes for each connection.
	% 	% \item Handled file metadata with \textbf{struct-based binary headers}, manually reassembled byte streams on server.
	% \end{itemize}
	
% 	\customItem[
% 		title=\href{https://github.com/AshkanArabim/OOP-project-1}{CLI Car Dealership \faExternalLink},
% 		duration=Team of 3 | April 2024,
% 		% keyHighlight=
% 	]
% 	\begin{itemize}
% 		\vspace{-0.5em}
% 		\itemsep -6pt {}
% %		\item Wrote an Object-Oriented car dealership software in \textbf{Java}, using inheritance, polymorphism, and interfaces.
% 		\item Wrote an Object-Oriented car dealership software in \textbf{Java} following MVC architecture \& object oriented design patterns. % with 2300+ lines of code.
% 		\item Used \textbf{Git \& GitHub} features such as pull requests, merges and branches to work in team of 3.
% %		\item Used Git \& GitHub features such as pull requests, merges, branches, and rebasing for teamwork in group of 3.
% %		\item Drafted overall software design in UML use-case and class diagrams.
% %		\item Followed Agile methodology to design, plan, and develop three phases of feature additions.
% %		\item Included operations to buy{,} restock{,} or add cars{,} monitor the revenue of each vehicle type{,} add / remove users, etc.
% %		\item Included automatic
% 	\end{itemize}
	
%	\customItem[
%		title=\href{https://github.com/AshkanArabim/gtk-timer}{Linux Timer \faExternalLink},
%		duration=January 2024,
%		% keyHighlight=
%	]
%	\begin{itemize}
%		\vspace{-0.5em}
%		\itemsep -6pt {}
%		\item Developed a \textbf{Linux} timer application using \textbf{GTK \& Vala}, using an event-driven architecture.
%		\item Implemented functionalities for starting, pausing, resetting, and editing timers, using different GTK4 widgets \& signals.
%	\end{itemize}
	
% 	\customItem[
% 		title=\href{https://github.com/AshkanArabim/pong-msp430}{Pong for MSP430 \faExternalLink},
% 		duration=Solo | November 2023,
% 		% keyHighlight=
% 	]
% 	\begin{itemize}
% 		\vspace{-0.5em}
% 		\itemsep -6pt {}
% 		%		\item Implemented smooth graphics using interrupts to avoid graphical bottlenecks.
% 		\item Designed and implemented a \textbf{C} Pong game for MSP430, with paddle movement, ball physics, and score tracking.
% %		\item Implemented interrupt-driven input handling for buttons, ensuring responsive gameplay with 30+ FPS.
% 		\item Achieved \textbf{30+ FPS} gameplay by using partial framebuffer updating instead of redrawing whole screen.
% 		\item Integrated buzzer audio feedback for game events such as ball-wall collisions and score updates.
% 	\end{itemize}
	
%   vv not including hacktoberfest 2023 because it was mostly wasted on translation projects...
	
%	\customItem[
%	title=Hacktoberfest 2023,
%	duration=October 2023,
%	% keyHighlight=
%	]
%	\begin{itemize}
%		\vspace{-0.5em}
%		\itemsep -6pt {}
%		\item Contributed to three open-source repositories through bug-fixes and translations.
%		\item Fixed subtitle cutoff bug in ASCII video player written in \textbf{C} by debugging the subtitle buffer. \href{https://github.com/aidancrowther/ASCIIPlay}{\faExternalLink}
%	\end{itemize}

	% \customItem[
	% title=\href{https://github.com/AshkanArabim/advent-of-code-2022}{Advent of Code 2022 - Annual Programming Challenge \faExternalLink},
	% duration=Solo | August 2023,
	% % keyHighlight=
	% ]
	% \begin{itemize}
	% 	\vspace{-0.5em}
	% 	\itemsep -6pt {}
	% 	% \item Coded \textbf{C++} solutions to 12 of 25 challenge questions using backtracking, graph traversal algorithms, and more.
	% 	\item Coded \textbf{C++} solutions to 12 of 25 data structures \& algorithms challenges.
	% 	\item Used classes, queues, vectors, and streams to efficiently calculate results based on given inputs.
	% \end{itemize}

    % \customItem[
    %     title=\href{https://github.com/AshkanArabim/email-classifier}{Adaptive Phishing Email Training System \faExternalLink},
    %     duration=August 2023 - Present,
    %     keyHighlight=Research project to develop dynamic software to train users' weaknesses in spotting phishing emails.
    % ]
    % \begin{itemize}
    %     \vspace{-0.5em}
    %     \itemsep -6pt {}
    %     \item Achieved 96\% phishing email detection accuracy to classify unlabeled emails for user training. (Keras, Python)
    % \end{itemize}

    % \customItem[
    %     title=Music Genre Classification,
    %     duration=Spring 2023,
    %     keyHighlight=Developed convolutional model to classify music genres based on their mel spectrogram.
    % ]
    % \begin{itemize}
    %     \vspace{-0.5em}
    %     \itemsep -6pt {}
    %     \item Led team of 3 to meet project deadlines by dividing tasks based on skill level (Project Management)
    %     \item Designed, tested, and implemented CNN model for classification (TensorFlow, Numpy, Conda)
    % % in case you need more:
    %% original linkedin title: AI4ALL College Pathways Participant
    % %- Led team of 3 to meet project deadlines by dividing tasks based on skill level.
    %%- Developed convolutional model to classify music genres based on their Mel spectrogram.
    %% - Learned about the applications and fundamental technical concepts of AI including the types of machine learning techniques and neural networks. 
    %% - Investigated ethical implications related to data processing and AI implementation 
    %% - Collaborated with peers on a project to preditct sudents' academic performance in college based on their background.
    % \end{itemize}

    % sample experience item: ----------------
    % \experienceItem[
    % company=Arizona State University,
    % location=Tempe{,} AZ,
    % position=Tutor (10 hours/week),
    % duration=Aug 2018 – May 2019
    % ]
    % \begin{itemize}
    %     \vspace{-0.5em}
    %     \itemsep -6pt {}
    %     \item Tutored 10-15 undergraduate engineering students per week in MATLAB programming and math coursework
    % \end{itemize}
    % ---------------------------------------------
    
%    \customItem[
%        title=\href{https://github.com/AshkanArabim/cybertweet-topics/tree/main}{Tweet topic detection \faExternalLink},
%        duration=August 2023,
%        % keyHighlight=
%    ]
%    \begin{itemize}
%        \vspace{-0.5em}
%        \itemsep -6pt {}
%        \item Tuned \textbf{Keras} model to detect one of 8 cyber-security topics mentioned in Tweets.
%        \item Utilized Twitter pre-trained word embeddings and Bidirectional GRU to achieve 93\% accuracy.
%    \end{itemize}
    
    % \customItem[
    %     title=Tweet Sentiment Classification,
    %     duration=Summer 2023,
    %     keyHighlight=Built almost perfect model to classify positive and negative Tweets
    % ]
    % \begin{itemize}
    %     \vspace{-0.5em}
    %     \itemsep -6pt {}
    %     \item Achieved 99\% accuracy using standard NLP preprocessing and along with Embedding and LSTM layers. (Keras)
    % \end{itemize}

	% \customItem[
	% 	title=Home Server,
	% 	duration=Solo | August 2023 - February 2024, % these dates aren't exact
	% 	% keyHighlight=
	% ]
	% \begin{itemize}
	% 	\vspace{-0.5em}
	% 	\itemsep -6pt {}
	% 	\item Built a personal \textbf{Ubuntu Server} to use for personal backups, self-hosting, file sharing, and GPU workloads.
	% 	\item Configured \textbf{12TB RAID5 array} with redundancy for reliable data backup and archiving. % with rsync?
	% 	% \item Created remote-accessible ML environment using an RTX 3060 GPU and 64GB RAM to use for research and \textbf{self-hosted LLMs}.
	% \end{itemize}
    
%    \customItem[
%        title=\href{https://ashkan.zone/}{Personal Portfolio \faExternalLink},
%        duration=Spring 2023,
%        % keyHighlight=Built complete portfolio website with React.JS and vanilla CSS. See https://ashkan.zone.
%    ]
%    \begin{itemize}
%        \vspace{-0.5em}
%        \itemsep -6pt {}
%        % \item Designed and implemented simple responsive and interactive UI using only vanilla CSS. (HTML, CSS)
%%        \item Built portfolio website with \textbf{React.JS} components and vanilla \textbf{CSS}.
%%       \item Used React.JS components to create single-page application, maximizing responsiveness. (React.JS, JavaScript)
%       \item Created a modern, minimal personal portfolio using \textbf{React.JS}, React Router, and some custom CSS.
%    \end{itemize}
    
    % \customItem[
    %     title=Chess Knight Path Finder,
    %     duration=Spring 2023,
    %     keyHighlight=Used DFS graph traversal to find the most optimal path for chess knight.
    % ]
    % \begin{itemize}
    %     \vspace{-0.5em}
    %     \itemsep -6pt {}
    %     \item Given a starting and ending point, algorithm finds least number of steps to reach destination (JavaScript)
    % \end{itemize}
    
    % \customItem[
    %     title=NPM dom package...
    % ]
    % \customItem[
    %     title=Weather Web App,
    %     duration=Spring 2023,
    %     keyHighlight=Used variety of web APIs to develop weather forecast web app with dynamic UI elements
    % ]
    % \begin{itemize}
    %     \vspace{-0.5em}
    %     \itemsep -6pt {}
    %     \item Used asynchronous programming to fetch and render data from OpenWeatherMap.org (JavaScript)
    %     \item Bundled site assets using Webpack to minimize deployment size (Webpack)
    % \end{itemize}
    
%    \customItem[
%        title=\href{https://github.com/AshkanArabim/todolist}{To Do List Web App \faExternalLink},
%        duration=Fall 2022,
%        % keyHighlight=
%    ]
%    \begin{itemize}
%        \vspace{-0.5em}
%        \itemsep -6pt {}
%        \item Developed interactive to-do list web app with local save function through vanilla \textbf{HTML{,} CSS \& JS}.
%              % \item Developed all internal logic from scratch following OOP principles % (JavaScript)
%        \item Used Chrome's local storage API for saving user data. % (JavaScript)
%        \item Followed typical \textbf{git/GitHub} version control workflow during implementation. % (git, GitHub)
%    \end{itemize}
    
%         \customItem[
%         title=\href{https://github.com/AshkanArabim/tic-tac-toe}{Tic Tac Toe Web Application \faExternalLink},
%         duration=Fall 2022,
%%         keyHighlight=Minimal Tic-Tac-Toe game with human-human{,} human-bot{,} and bot-bot game-modes.
%         ]
%         \begin{itemize}
%             \vspace{-0.5em}
%             \itemsep -6pt {}
%             % \item Implemented smooth animations using CSS keyframes to improve site nativation. (CSS)
%%             \item Utilized CSS glow{,} shading{,} and animations for a modern and sleek look. (HTML, CSS)
%%             \item Followed OOP principles to track game progress. (JavaScript)
%%             \item Used typical git/GitHub workflow during implementation (git, GitHub)
%%             ------------------------------ OLD BULLETS, DON'T USE ^^ ---------------------------------------------
%             \item Developed a web-based tic-tac-toe application using \textbf{HTML, CSS, and JavaScript}, with a special focus on the visuals.
%             \item Used JavaScript to manage game state, handle player moves, and determine game outcomes.
%%             \item Utilized CSS glow{,} shading{,} and animations for a Neumorphist UI style.
%         \end{itemize}

%    \customItem[
%        title=\href{https://github.com/AshkanArabim/blabber}{Blabber {-} a CLI Twitter replica \faExternalLink},
%        duration=Fall 2022,
%        % keyHighlight=Made CLI app that allows users to create an account{,} post{,} follow others{,} see a timeline{,} and delete their account
%    ]
%    \begin{itemize}
%        \vspace{-0.5em}
%        \itemsep -6pt {}
%        \item Made \textbf{Java} CLI app for users to create an account{,} post{,} follow others{,} see a timeline{,} and delete their account.
%        \item Used scanners and writers to save{,} update{,} and delete user information and posts. % (Java)
%    \end{itemize}

    % \customItem[
    %     title=Contribution to Monkeytype,
    %     duration=Spring 2021,
    %     keyHighlight=Added 3 levels of Persian tests to open source typing test website
    % ]
    % \begin{itemize}
    %     \vspace{-0.5em}
    %     \itemsep -6pt {}
    %     \item Developed script to extract and check words from large text bodies using Vajehyab.com dictionary API (Python)
    %     \item Uploaded more than 21,000 most used Persian words to site database (git, GitHub)
    % \end{itemize}
    
\end{workSection}

\begin{workSection}{Leadership}
    % \customItem[
    %     title=CS Student Council,
    %     %         keyHighlight=One of the founders{,} and current member of the Council{,} helping students voice their concerns to the administration.
    %     keyHighlight=Empowered the CS students to voice their suggestions by providing feedback channels and hosting town halls.,
    %     duration=Fall 2023 - Present
    % ]
    % \begin{itemize}
    %     \vspace{-0.5em}
    %     \itemsep -6pt {}
    %     % \item Drafted ...
    % \end{itemize}
    
    \customItem[
    	title=President \& Founder - \href{https://www.instagram.com/foss.utep/}{\textbf{Free and Open Source Software Club at UTEP} \faExternalLink},
    	duration=December 2023 - March 2025
    ]
    \begin{itemize}
    	\vspace{-0.5em}
    	\itemsep -6pt {}
%    	\item Encouraged participation in open-source projects through workshops, info sessions, competitions, and social events.
    	\item Scaled UTEP's first open-source-focused student organization from \textbf{0 to 264 members in 3 semesters}. % see post-resignation stats
		\item Organized \textbf{47 workshops, personally presenting 20}, covering topics such as Git, Linux, Kubernetes, Vim, Emacs, and more.
    	% \item Led team of 6 officers to host weekly workshops on git, Linux, Vim, open-source software development, and more.
		\item Hosted UTEP's \textbf{first open-source hackathon}, OpenHack, attended by \textbf{12+ first-time contributors}.
    \end{itemize}

%    \customItem[
%        title=Treasurer - Association for Computing Machinery at UTEP,
%        %        keyHighlight=Multiple roles{,} including publicity officer \& treasurer,
%        % keyHighlight=Promoted side-projects and research by hosting informational and technical workshops to 40+ students each semester.,
%        duration=August 2022 - January 2024
%    ]
%    \begin{itemize}
%        \vspace{-0.5em}
%        \itemsep -6pt {}
%%        \item Planned and executed the Sun City Hackathon; a three-day competition attended by \textbf{more than 20 students} to develop novel AI-powered apps.
%        \item Held the Sun City Hackathon; a three-day competition attended by \textbf{more than 20 students} to develop AI-powered apps.
%%%        \item Promoted side-projects and research by hosting informational and technical workshops to 40+ students each semester.
%%              % \item Recruited 50+ members through public presentations. # bad
%%              % \item Engaged with the student body to recruit more than 50 students.
%%              % \item Organized and planned of more than 6 workshops sponsored by companies such as Google.
%%              % \item Presented multiple interactive workshops attended by org members on topics such as Node.JS.
%%              %        \item Promoted research and open-source contributions by hosting info sessions and technical workshops to 40+ students each semester.
%    \end{itemize}
\end{workSection}

\skills{
	% note: NOBODY CARES ABOUT SKILL RATINGS!!!
	\skillItem[
		%category=Programming / Markup Languages,
		category=Languages,
		%Python: bee using heavily since Jan 2022
		%Vala: since December 2023
		%HTML/CSS: heavily since Jan 2022
		%JS: Heavily since May 2022
		%Java: since August 2022
		%Bash: heavily since June 2023
		%Haskell: used in PL, Jan - April 2024
		%LaTeX: heavily in June
		%C: heavily in Org / Hacktoberfest, August - December 2023
		skills=
		Python{,} %(adv.){,} 
		Java{,} %(adv.){,} 
		JavaScript{,} %(adv.){,} 
		TypeScript{,} %(adv.){,} 
		SQL{,} %(med.){,} 
		Dart{,} %(med.){,} 
		Vala{,} %(med.){,} 
		HTML/CSS{,} %(med.){,} 
		% UML{,} %(med.){,} 
		Bash{,} %(med.){,} 
		Haskell{,} %(med.){,} 
		LaTeX{,} %(novice){,} 
		% C++{,} %(novice){,} 
		% C{,} %(novice){,} 
		% to be fair, I used C++ more in-depth than C. plus, it looks more dope
		% PHP{,} %(novice){,} 
	]
	\\
	\skillItem[
		category=Libraries,
		% for data science / ML:
		skills=
		%GTK(med.){,} 
		ReactJS{,} %(adv.){,} 
		PyTorch{,} %(med.){,} 
		TensorFlow{,} %(med.){,} 
		NumPy{,} %(med.){,} 
		Pandas{,} %(med.){,} 
		Django{,} %(med.){,}
		JUnit{,} 
		% Mockito{,}
		FastAPI{,} %(med.){,} 
		Supabase{,} %(med.){,} 
		Firebase{,} %(med.){,} 
		Flask{,} %(med.){,} 
		Redux{,} %(med.){,} 
		% Node.JS{,} %(med.){,} 
		% RabbitMQ{,} %(novice){,} 
		% PHP{,} %(novice){,} % I used this for the DB final project
		% MATLAB{,} %(novice){,} 
		% Vue.JS{,} %(novice){,} % plan B
		% Matplotlib{,} %(novice) 
	]
	\\
	\skillItem[
		category=Tools,
		skills=
		% Unix/Linux{,} %(adv.){,} 
		Linux{,} %(adv.){,} 
		Git{,} %(adv.){,} 
		GitHub{,} %(adv.){,} 
		%GitLab{,} %(med.){,} 
		% RDBMS{,} %(adv.){,} 
		Docker{,} %(adv.){,} 
		Docker Compose{,} %(med.){,} 
		% Docker Swarm{,} %(med.){,} 
		% Observe{,} %(med.){,}
		AWS{,} %(novice){,}
		GCP{,} %(novice){,}
		Kubernetes{,} %(novice){,} 
		% Make{,} %(novice){,} 
		% Snowflake{,} %(novice){,}
		New Relic{,} %(novice){,}
		MongoDB(NoSQL){,} 
		PostgreSQL{,} %(med.){,} 
		MySQL{,} %(med.){,} 
		% Bazel{,} %(novice){,} % plan B
		% imagemagick 
		% Conda{,} 
		% Other stuff: -----------------
		% Microsoft Office{,} 
		% Adobe Photoshop{,} 
		%Adobe Illustrator{,} 
		% nah: -----------------------
		% SSH{,} Singularity{,} Webpack {,} Jupyter,
		% DON'T INCLUDE:
		% Webpack
	]
	%\\
	%\skillItem[
	%    category=Certifications,
	%    skills=Coursera Deep Learning Specialization {-} October 2023
	%]
	% \\
	% \skillItem[
	%     category=Foreign Languages,
	%     skills=Persian (Native){,} Spanish (Fluent){,} French (Beginner)
	% ]
}

% \begin{workSection}{Publications}
% cite ACM Mobihoc paper
% \end{workSection}

\end{document}